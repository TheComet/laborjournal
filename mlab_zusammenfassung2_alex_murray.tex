\documentclass[notitlepage]{fhnwreport/fhnwreport}

\usepackage{gensymb}
\usepackage{textcomp}
\usepackage[T1]{fontenc}
\usepackage{lipsum}
\usepackage{titling}       % Abstract on titlepage
\usepackage{changepage}    % Change width of abstract text
\usepackage{float}
\usepackage{amsmath}
\usepackage{amssymb}
\usepackage{mathrsfs}
\usepackage{todonotes}
\usepackage[backend=bibtex,bibstyle=numeric,sorting=nyt]{biblatex}
\usepackage[toc,page]{appendix}
\usepackage{siunitx}
\usepackage{subcaption}
\usepackage{pdfpages}
\sisetup{output-exponent-marker=\ensuremath{\mathrm{E}}}

% Labels per section
\numberwithin{equation}{section}

% For MATLAB code
\usepackage{listings}
\usepackage{color} %red, green, blue, yellow, cyan, magenta, black, white
\definecolor{mygreen}{RGB}{28,172,0} % color values Red, Green, Blue
\definecolor{mylilas}{RGB}{170,55,241}

% Also required for tuning appearance of MATLAB code
\usepackage{microtype}
\usepackage[scaled]{beramono}
\newcommand\smaller{\fontsize{8}{8.2}\selectfont}
\newcommand*\LSTfont{\smaller\ttfamily\SetTracking{encoding=*}{-60}\lsstyle}

% Laplace L
\DeclareMathOperator{\laplace}{\mathscr{L}}



\begin{document}

\section*{Zusammenfassung Synthese von kanonischen Reaktanzeintoren}

Sei eine  \"Ubertragungsfunktion  eines  LC-Filters bekannt, sollen die L- und
C-Elemente numerisch berechnet werden. Diese ``synthetisierung'' erfolgt durch
bestimmung        der       Leerlaufeingangsreaktanz       und/oder        die
Kurzschlusseingangsreaktanz  des  Netzwerkes  aus   der   \"Ubergangsfunktion.

Mit   diesen   sogenannten   Reaktanzfunktionen   werden   anschliessend   die
Netzwerkelemente berechnet. Bei der Verwendung von sogenannten Tabellenfiltern
sind alle diese Schritte vorher schon durchgef\"uhrt worden  (offline) und die
normierten Werte der Elemente findet  man in ``Tabellen'' (fr\"uher verwendete
man   Papier,   heute   sind   sie   elektronisch  abgelegt).  Damit   spielen
Reaktanzeintore  (RET)  und  Reaktanzfunktionen  eine  wichtige  Rolle  in der
Netzwerktheorie und in der Synthese von LC-Filtern.

Zu einer LC-Impedanz-  oder  Admittanzfunktion  mit  der h\"ochsten Potenz $n$
geh\"oren Schaltungen mit \textit{mindestens}  $n$  Elementen. Das heisst dass
Schaltungen auch mit mehr Elementen dieselbe  Reaktanzfunktion haben k\"onnen.
Solche Schaltungen sind als  RET  bekannt  (Reaktanzeintore).  Liegen  nur $n$
Elemente vor,  so ist die Schaltung als MRET (Minimum-Reaktanzeintor) bekannt.
Eine   interessante   Eigenschaft   von   MRETs   ist,   dass   die  Zahl  der
Induktivit\"aten gegen\"uber der  Kapazit\"aten  h\"ochstens  um 1 verschieden
sein kann. Ist dies  nicht  der  Fall, dann ist die Schaltung ein RET und kein
MRET. Jede RET kann zu einer MRET reduziert werden. Dabei geht  man  wie folgt
vor:

\begin{itemize}
\item  Man  sucht   bei   \textit{offenen}   Klemmen   alle  Kreise,  die  nur
Kapazit\"aten oder  nur  Induktivit\"aten  enthalten  (C-Kreise und L-Kreise).
\item Man reduziert das Netzwerk,  indem  sukzessiv  Elemente  in  den  L- und
C-Kreisen  weggelassen werden. Dabei d\"urfen  beim  weglassen  keine  anderen
Zweige stromlos werden.
\item   Nun   schliesst  man  die  Klemmen  kurz  und  sucht  nach   C-   oder
L-Trennb\"undel.  Existieren solche, so schliesst man Elemente kurz, bis keine
C- oder L-Trennb\"ugel mehr  existieren. Dabei d\"urfen keine anderen Elemente
kurzgeschlossen werden.
\end{itemize}

Besitzt das Netzwerk keine Kreise und Trennb\"undel mehr, die aus nur L oder C
bestehen, so ist es minimal in seiner Anzahl von Elementen.

Reaktanzeintore lassen sich in vier Typen einteilen (L-Typ,  C-Typ,  P-Typ und
S-Typ)  und  jeder  dieser  Typen  l\"asst  sich  mit  vier  sog.  kanonischen
Schaltungen  realisieren.  L-Typen sind bei $\omega=0$ ein Kurzschluss und bei
$\omega\to\infty$ ein Unterbruch.  C-Typen  sind bei $\omega=0$ ein Unterbruch
und    bei    $\omega\to\infty$    ein    Kurzschluss.    P-Typen   sind   bei
$\omega=0,\infty$  ein  Kurzschluss  und  S-Typen  sind  bei $\omega=0,\infty$
ein Unterbruch. Die zugeh\"origen Reaktanzfunktionen sind rational  gebrochene
Funktionen  in  s   mit  ganz  bestimmten  Eigenschaften.  Zum  Auffinden  der
kanonischen  Schaltungen  und  zur  Berechnung  der  Elemente  ben\"otigt  man
Methoden/Verfahren wie die Partialbruchzerlegung und die Kettenbruchzerlegung.
Bei  diesen  numerischen Verfahren ist die Genauigkeit (signifikante  Stellen)
sehr wichtig und  stellt  ein zentrales Thema dar. Vor allem ist die ``robuste
und genaue'' Nullstellensuche von Polynomen (root  finding)  eine  sehr grosse
Herausforderung. In verschiedenen Tools wie Matlab,  Maple,  usw.  stehen  zum
Teil Befehle zur Verf\"ugung, welche  eine  Partial- oder Kettenbruchzerlegung
durchf\"uhren.

Die Erste kanonische Schaltung  besteht  aus  einer  Kette  von LC-Gliedern in
Serie geschaltet. Je nach  Typ (L, C, P oder S) ist zus\"atzlich auch noch ein
C oder L in Serie mit der Kette geschaltet  (bei  P oder S eben nicht). Um die
einzelnen   LC    Elemente    zu    berechnen    wird   die   Impedanzfunktion
partialbruchzerlegt. Dies gibt uns die  Polstellen  der  einzelnen LC-Glieder.
Die Nullstellen k\"onnen berechnet werden, indem das Netzwerk in sein ``Duales
Gegenst\"uck''  transformiert wird.  Das  erfordert  nichts  anderes  als  die
Z\"ahler   und   Nenner   zu   tauschen.   Die   Partialbruchzerlegung  dieser
Admittanzfunktion ergibt die Nullstellen. Aus diesen Resultaten  schliesst man
auf die einzelen L's und C's.

Die  zweite  Art   von   kanonische   Schaltung   besteht   aus   mehreren  in
abwechslungsweise   serie-   und   parallelgeschalteten   L's  und  C's.   Die
Impedanzfunktion  bzw.  die Admittanzfunktion des  dualen  Gegenst\"ucks  wird
diesmal   Kettenbruchzerlegt,  um  die  Pol  und  Nullstellen  zu   bestimmen.

Es wurden 5 verschiedene Impedanzfunktionen in der Aufgabe gegeben. F\"ur jede
Impedanzfunktion sollen die  4  kanonischen Schaltungen berechnet werden. Dies
wurde in MATLAB implementiert und mit dem P2 Java-Tool verglichen. MATLAB kann
mittels \textit{residue()} eine Partialbruchzerlegung  berechnen.  Leider kann
MATLAB  aber  kein  Kettenbruch   berechnen.   Jedoch  habe  ich  online  eine
Implementation gefunden.

Bei der ersten  Aufgabe  ergibt eine Kettenbruchzerlegung der Impedanzfunktion
nur ganze Zahlen ($C_{\infty}=5, L_1=4, C_1=3, L_2=2, C_2=1$).

Bei der zweiten Aufgabe ergibt die Partialbruchzerlegung der Admittanzfunktion
nur ganze Zahlen ($L_1=6, C_1=5, L_2=4, C_2=3, L_3=2, C_3=1$).

Bei der dritten  Aufgabe ergibt die Partialbruchzerlegung der Impedanzfunktion
nur ganze  Zahlen  ($L_{\infty}=4,  C_1=4,  L_1=3, C_2=3, L_2=2, C_3=2, L_3=1,
C_4=1$).

Bei  der vierten Aufgabe ergibt die Kettenbruchzerlegung der  Impedanzfunktion
nur ganze Zahlen ($C_{\infty}=1,  C_1=2,  L_1=1, C_2=3, L_2=2, C_3=4, L_3=4$).

Bei der f\"unften Aufgabe ergibt die Kettenbruchzerlegung der Impedanzfunktion
gute Zahlen ($L_{\infty}=1e-5,  C_1=2.5e-10,  L_1=2e-5,  C_2=1e-10,  L_2=3e-5,
C_3=5e-11$).

Die Polstellen und Nullstellen in der  f\"unften  Aufgabe  sind  recht  extrem
gew\"ahlt.  MATLAB  hatte  bei der Partialbruchzerlegung ein bisschen  M\"uhe,
bzw.  er  hat  zwei  Polstellen und Zwei  Nullstellen  auf  den  gleichen  Ort
platziert,   obwohl   sie  eigentlich   nicht   gleich   sein   sollten.   Die
Kettenbruchzerlegung hingegen hat gut funktioniert.

In diesem Versuch habe ich zum ersten mal die Kettenbruchzerlegung im ``echten
Welt''  einsetzen  k\"onnen.  Ich  dachte  bisher,  dass  es   nicht  wirklich
Anwendungen  hatte.  Es  ist immer  interessant,  wenn  die  Limitationen  des
verwendeten  Rechentools  \"uberschritten  werden  (hier:  Bei  der  f\"unften
Aufgabe). Ich denke dabei  immer  wieder  nach, wie viele Stunden verschwendet
werden k\"onnten wenn so etwas nicht gleich auff\"allt.

\end{document}

