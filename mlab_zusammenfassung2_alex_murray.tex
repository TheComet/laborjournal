\documentclass[notitlepage]{fhnwreport/fhnwreport}

\usepackage{gensymb}
\usepackage[T1]{fontenc}
\usepackage{textcomp}
\usepackage{booktabs,caption,fixltx2e}
\usepackage[flushleft]{threeparttable}
\usepackage{float}
\usepackage{amsmath}
\usepackage[style=mla,backend=bibtex]{biblatex}
\usepackage[toc,page]{appendix}
\usepackage{siunitx}
\usepackage{pdfpages}
\usepackage{pgf}
\sisetup{output-exponent-marker=\ensuremath{\mathrm{E}}}


\begin{document}

\section*{Zusammenfassung Synthese von kanonischen Reaktanzeintoren}

Sei eine  \"Ubertragungsfunktion  eines  LC-Filters bekannt, sollen die L- und
C-Elemente numerisch berechnet werden. Diese ``synthetisierung'' erfolgt durch
bestimmung        der       Leerlaufeingangsreaktanz       und/oder        die
Kurzschlusseingangsreaktanz  des  Netzwerkes  aus   der   \"Ubergangsfunktion.

Mit   diesen   sogenannten   Reaktanzfunktionen   werden   anschliessend   die
Netzwerkelemente berechnet. Bei der Verwendung von sogenannten Tabellenfiltern
sind alle diese Schritte vorher schon durchgef\"uhrt worden  (offline) und die
normierten Werte der Elemente findet  man in ``Tabellen'' (fr\"uher verwendete
man   Papier,   heute   sind   sie   elektronisch  abgelegt).  Damit   spielen
Reaktanzeintore  (RET)  und  Reaktanzfunktionen  eine  wichtige  Rolle  in der
Netzwerktheorie und in der Synthese von LC-Filtern.

Reaktanzeintore lassen sich in  vier  Typen einteilen (L-Typ, C-Typ, P-Typ und
S-Typ)  und  jeder  dieser  Typen  l\"asst  sich  mit  vier  sog.  kanonischen
Schaltungen  realisieren.  Die  zugeh\"origen Reaktanzfunktionen sind rational
gebrochene  Funktionen  in  s mit ganz bestimmten Eigenschaften. Zum Auffinden
der kanonischen Schaltungen und  zur  Berechnung  der  Elemente ben\"otigt man
Methoden/Verfahren wie die Partialbruchzerlegung und die Kettenbruchzerlegung.
Bei diesen  numerischen  Verfahren  ist die Genauigkeit (signifikante Stellen)
sehr  wichtig und stellt ein zentrales Thema dar. Vor allem ist die  ``robuste
und genaue'' Nullstellensuche von Polynomen (root  finding)  eine  sehr grosse
Herausforderung. In  verschiedenen  Tools  wie  Matlab, Maple, usw. stehen zum
Teil Befehle zur  Verf\"ugung,  welche eine Partial- oder Kettenbruchzerlegung
durchf\"uhren.



\end{document}

