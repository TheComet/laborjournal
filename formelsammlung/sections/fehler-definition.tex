\section{Fehler}

\subsection{Fehlertypen}

Es ist zwischen den folgenden Fehlertypen zu unterscheiden.
\begin{description}
    \item[Systematische Fehler] sind kontrollierte Fehler bzw. systematische Unsicherheiten. Sie sind von
    der Umgebung oder vom Messvorgang selbst verursacht und verursachen entweder eine systematische
    Abweichung des Messergebnisses vom ``wahren Wert'' oder lediglich eine Unsicherheit der Messgr\"osse.
    Beispiele sind Massst\"abe (messen nur bein einer bestimmten Temperatur richtig), Fehler von Messinstrumenten,
    Eichunsicherheiten oder Eichfehler von Messinstrumenten, oder verborgene, \"aussere Einfl\"usse auf das
    Experiment.
    \item[Zuf\"allige Fehler] sind unkontrollierte Fehler die sich bei jeder Messung \"andern. Sie verursachen,
    dass die Messwerte bei mehrmaliger Messung statistisch um den ``wahren Wert'' schwanken. Beispiele
    sind Rauschen, Streuung, Genauigkeit der Sinnesorgane des Beobachters, Geb\"audesch\"utterungen,
    Luftstr\"ome usw.
\end{description}


\subsection{Genauigkeit}

Die Anzahl der Ziffer muss \"ubereinstimmen, z.B. $T=(14.38\pm0.15)$ oder $T=(100\pm2)$ und \emph{nicht}
$T=(14.36\pm2)$.

Zuf\"allige Fehler werden mit $s$ gekennzeichnet, Unsicherheiten und Absch\"atzungen werden mit $\Delta$
gekennzeichnet.

Beispiele:
\begin{itemize}
    \item $\bar{T} = 15.6 \textrm{s}$ (mittelwert)
    \item $s_T = 0.5 \textrm{s}$ (absoluter Fehler)
    \item $r_T = \frac{s_T}{T} = \textrm{\%}$ (relativer Fehler)
\end{itemize}

Relative Fehler k\"onnen mit \%, \textperthousand, oder ppm angegeben werden.


