\section{Fehlerfortpflanzung}

\subsection{Gesetz von Gauss}

In der Mehrzahl der F\"alle liefern physikalische Versuche nicht unmittelbar das gesuchte Messergebnis,
sondern die Resultatgr\"osse $R$ ist eine Funktion von mehreren Gr\"ossen $R(x, y, z, \ldots)$
wobei
\begin{center}
    \begin{tabular}{c}
        $x = \bar{x} \pm s_{\bar{x}}$ \\
        $y = \bar{y} \pm s_{\bar{y}}$ \\
        $z = \bar{z} \pm s_{\bar{z}}$ \\
    \end{tabular}
\end{center}

Den Mittelwert $\bar{R}$ erh\"alt man, indem man einfach die Mittelwerte der verschiedenen Messgr\"ossen
in der Funktion $R$ einsetzt.
\begin{equation}
    \bar{R} = R(\bar{x}, \bar{y}, \bar{z}, \ldots)
\end{equation}

Der mittlere, absolute Fehler $S_{\bar{R}}$ ergibt sich aus dem \emph{Gauss'schen Fehlerfortpflanzungsgesetz}:
\begin{equation}
    S_{\bar{R}} = \sqrt{  \left( \left.\frac{\partial R}{\partial x} \right\rvert_{\bar{R}} \cdot s_{\bar{x}} \right)^2
                        + \left( \left.\frac{\partial R}{\partial x} \right\rvert_{\bar{R}} \cdot s_{\bar{x}} \right)^2
                        + \left( \left.\frac{\partial R}{\partial x} \right\rvert_{\bar{R}} \cdot s_{\bar{x}} \right)^2
                        + \ldots }
\end{equation}


\subsection{H\"aufige F\"alle}

\subsubsection{Addition und Subtraktion}

Sie $R=x+y$ oder $R=x-y$ so ergibt sich wegen $\frac{\partial R}{\partial x} = 1$ und $\frac{\partial R}{\partial y} = 1$
nach Einsetzen ins Fehlerfortpflanzungsgesetz
\begin{equation}
    s_{\bar{R}} = \sqrt{ s_{\bar{x}}^2 + s_{\bar{y}}^2 }
\end{equation}


\subsubsection{Multiplikation und Division}

Sei $R=x \cdot y$ oder $R=\frac{x}{y}$ so ergeben die partitiellen Ableitungen $\frac{\partial R}{\partial x} = y$
resp. $\frac{\partial R}{\partial y} = x$.

nach Einsetzen ins Fehlerfortpflanzungsgesetz
\begin{equation}
    s_{\bar{R}} = \sqrt{ (y \cdot s_{\bar{x}})^2 + (x \cdot s_{\bar{y}})^2 }
    \hspace{10mm} \textrm{resp.} \hspace{10mm}
    s_{\bar{R}} = \sqrt{ \left( \frac{1}{y} \cdot s_{\bar{x}} \right)^2 + \left( \frac{x}{y^2} \cdot s_{\bar{y}} \right)^2 }
\end{equation}

F\"ur den relativen Fehler von $R$ gilt bei beiden Operationen
\begin{equation}
    r_R = \frac{s_{\bar{R}}}{R} = \sqrt{ \left( \frac{s_{\bar{x}}}{x} \right)^2 + \left( \frac{s_{\bar{y}}}{y} \right)^2 } = \sqrt{ r_x^2 + r_y^2 }
\end{equation}


\subsubsection{Potenzen}

Es sei die gemessene Gr\"osse $R=x^n$, so ergibt die partitielle Ableitung $s_R = n \cdot x^{n-1} \cdot s_x = n \frac{s_x}{x} x^n$
und es folgt f\"ur den relativen Fehler
\begin{equation}
    r_R = \frac{s_R}{R} = n \cdot r_x
\end{equation}

Dies bedeutet dass z.B. beim Wurzelziehen der relative Fehler kleiner wird, beim Quadrieren wird
er verdoppelt. Will man z.B. die Boltzmannkonstante $\sigma$ aus der spezifischen Emission und der
Temperatur eines Hohlraumstrahlers bestimmen ($M_s = \sigma T^4$), so bewirkt ein relativer
Fehler von 5\% in der Temepraturmessung einen Fehler von 20\% in $\sigma$!

