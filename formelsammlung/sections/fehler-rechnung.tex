\section{Fehlerrechnung}

\subsection{Unter gleichen Bedinungen widerholt gemessene Gr\"ossen}

Arithmetischer Mittelwert
\begin{equation}
    \bar{x} = \frac{1}{N} \sum_{i=1}^{N} x_i
\end{equation}

Standardabweichung
\begin{equation}
    s_x = \sqrt{ \frac{ \sum_{i=1}^{N} (x_i - \bar{x})^2 }{ N-1 } }
\end{equation}

Fehler des Mittelwertes
\begin{equation}
    s_{\bar{x}} = \frac{ s_x }{ \sqrt{N} }
\end{equation}


\subsection{Mittelwertbildung mit Gewichten}

Durch verschiedene Messmethoden bestimmte Messgr\"ossen

\begin{center}
    \begin{tabular}{c}
        $x_1 = \bar{x_1} \pm s_{\bar{x_1}}$ \\
        $x_2 = \bar{x_2} \pm s_{\bar{x_2}}$ \\
        $\ldots$                            \\
        $x_n = \bar{x_n} \pm s_{\bar{x_n}}$ \\
    \end{tabular}
\end{center}

kann $\bar{x}$ und $s_{\bar{x}}$ folgendermassen bestimmt werden.

Den wahrscheinlichsten Wert von $\bar{x}$
\begin{equation}
    \bar{x} = \frac{ \sum_{i=1}^{N} g_{\bar{x_i}} \cdot x_i }{ \sum_{i=1}^{N} g_{\bar{x_i}} }
    \hspace{5mm} \textrm{mit} \hspace{5mm}
    g_{\bar{x_i}} = \frac{1}{s_{\bar{x_i}}^2}
\end{equation}

Fehler des gewogenen Mittels
\begin{equation}
    s_{\bar{x}} = \frac{1}{\sqrt{\sum_{i=1}^{N} g_{\bar{x_i}}}}
\end{equation}

