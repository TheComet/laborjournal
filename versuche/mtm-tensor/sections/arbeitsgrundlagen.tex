\section{Arbeitsgrundlagen}

\subsection{Tensor}

Ein  Tensor ist  ein  mathematisches  Objekt  das  f\"ur  die  Beschreibung  von
physikalischen Eigenschaften benutzt werden kann.  Tensoren  sind lediglich eine
Verallgemeinerung von Skalare Gr\"ossen  und Vektoren; Ein Skalar ist ein Tensor
nullter Ordnung und ein Vektor ist ein Tensor erster Ordnung.

Die  Ordnung  eines  Tensors  ist  definiert  durch die f\"ur  die  Beschreibung
notwendig  Anzahl  Richtungen.  Beispielsweise  k\"onnen  Objekte, die nur  eine
Richtung brauchen, vollst\"andig  durch  einen 3x1 Vektor beschrieben werden und
Objekte, die zwei Richtungen brauchen, k\"onnen mit 9 Zahlen in einer 3x3 Matrix
arrangiert beschrieben werden.

Das MTM kann durch ein Tensor zweiter Ordnung beschrieben werden.


\subsection{Das MTM im Hauptachsensystem}

Legen  wir  den Ursprung  unseres  Koordinatensystems  in  den  Schwerpunkt  des
Testk\"orpers und richten dieses parallel zu den MTM-Hauptachsen aus, so spricht
man vom Hauptachsensystem.

\begin{figure}[H]
    \center
    \includegraphics[width=.4\textwidth]{images/mtm-hauptachsensystem.pdf}
    \caption{}
\end{figure}

Eine Konvention ist, dass das gr\"osste MTM  bei  der  Rotation  um  die x-Achse
vorliegt und das kleinste MTM bei der  Rotation  um  die  z-Achse  vorliegt.  In
diesem  Koordinatensystem  (wenn  der  K\"orper  so   ausgerichtet  ist)  lautet

\begin{equation}
    \vec{\vec{I}} =
    \begin{pmatrix}
        I_{x,x} & I_{x,y} & I_{x,z} \\
        I_{y,x} & I_{y,y} & I_{z,z} \\
        I_{z,x} & I_{z,y} & I_{y,z} \\
    \end{pmatrix} =
    \begin{pmatrix}
        I_{x,x} & 0       & 0       \\
        0       & I_{y,y} & 0       \\
        0       & 0       & I_{z,z} \\
    \end{pmatrix} =
    2m \cdot
    \begin{pmatrix}
        e^2 & 0   & 0           \\
        0   & d^2 & 0           \\
        0   & 0   & (d^2 + e^2) \\
    \end{pmatrix}
\end{equation}


\subsection{MTM-Tensor eines verdrehten K\"orpers}

Verdreht man einen K\"orper um die x-Achse um den Winkel $\vartheta$ so entsteht folgender Tensor.

\begin{align}
    \vec{\vec{I}}^* &= \vec{\vec{R_x}} \cdot \vec{\vec{I}} \cdot \vec{\vec{R_x}}^{-1} \\
                    &=  \begin{pmatrix}
                            1 & 0               & 0                \\
                            0 & \cos{\vartheta} & -\sin{\vartheta} \\
                            0 & \sin{\vartheta} & \cos{\vartheta}  \\
                        \end{pmatrix} \cdot
                        \begin{pmatrix}
                            I_{x,x} & 0       & 0       \\
                            0       & I_{y,y} & 0       \\
                            0       & 0       & I_{z,z} \\
                        \end{pmatrix} \cdot
                        \begin{pmatrix}
                            1 & 0                & 0               \\
                            0 & \cos{\vartheta}  & \sin{\vartheta} \\
                            0 & -\sin{\vartheta} & \cos{\vartheta} \\
                        \end{pmatrix} \\
                    &=  \begin{pmatrix}
                            I_{x,x} & 0                                                         & 0                                                         \\
                            0       & I_{y,y}\cos^2{\vartheta} + I_{z,z}\sin^2{\vartheta}       & \frac{1}{2}\left(I_{y,y} - I_{z,z}\right)\sin{2\vartheta} \\
                            0       & \frac{1}{2}\left(I_{y,y} - I_{z,z}\right)\sin{2\vartheta} & I_{y,y}\sin^2{\vartheta} + I_{z,z}\cos^2{\vartheta}       \\
                        \end{pmatrix} \\
\end{align}

wobei $\vec{\vec{R_x}}$ die Rotationsmatrix um die x-Achse ist
\begin{equation}
    \vec{\vec{R_x}} =
    \begin{pmatrix}
        1 & 0               & 0                \\
        0 & \cos{\vartheta} & -\sin{\vartheta} \\
        0 & \sin{\vartheta} & \cos{\vartheta}  \\
    \end{pmatrix}
\end{equation}


\subsection{Verdrehter rotierender K\"orper}

Ein um die x-Achse verdrehter K\"orper wird mit konstanter Winkelgeschwindigkeit
$\omega$  um  die  z-Achse  rotiert.  Der  Tensor  $\vec{\vec{I}}^*$  muss  also
zeitabh\"angig   gedreht   werden.   Es   wird   zuerst   eine   zeitabh\"angige
Rotationsmatrix $R_z$ erstellt.
\begin{equation}
    \vec{\vec{R_z}}(t) =
    \begin{pmatrix}
        \cos{\omega t} & -\sin{\omega t} & 0 \\
        \sin{\omega t} & \cos{\omega t}  & 0 \\
        0              & 0               & 1 \\
    \end{pmatrix}
\end{equation}

Nun kann der zeitabh\"aniger Drehimpulsvektor beschrieben werden durch
\begin{equation}
    \vec{L}(t) = \vec{\vec{R_z}} \cdot \vec{\vec{I^*}} \cdot \vec{\vec{R_z}} \cdot
    \begin{pmatrix}
        0      \\
        0      \\
        \omega \\
    \end{pmatrix}
\end{equation}


