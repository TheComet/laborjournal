\section{Durchf\"uhrung}

\subsection{Versuchsanordnung Lichtgeschwindigkeitsmessung}

Der Laser  wurde eingeschaltet und  der Umlenkspiegel wurde so  angepasst, bis
der Laserstrahl parallel zur langen Tischkante verlaufte, was in der Abbildung
\ref{fig:laser-angle} links  ersichtlich ist. Weiter wurde die  H\"ohe $h$ vom
Laser zum Tisch gemessen und der  Strahl nach dem Umlenkspiegel an die gleiche
h\"ohe $h$  angepasst, was in der  Abbildung \ref{fig:laser-angle} illustriert
ist.

\begin{figure}[H]
    \center
    \includegraphics[width=.5\textwidth]{images/laser-angle.pdf}
    \caption{Tisch mit Laser von Oben (links) und von der Seite (rechts) gesehen}
    \label{fig:laser-angle}
\end{figure}

Der  Laser und  der Hohlspiegel  $HS$  befinden sich  auf separate  Tische. Die
H\"ohen der Tische zum Boden $h_3$ und $h_4$ wurden gemessen. Die H\"ohe $h_2$
-- also  die Distanz  zwischen dem  Zentrum des Hohlspiegels  und dem  Tisch --
wurde weiter durch  Feinjustieren des Umlenkspiegels angepasst  bis $h_1+h_3 =
h_2+h_4$.

\begin{figure}[H]
    \center
    \includegraphics[width=.8\textwidth]{images/laser-height.pdf}
    \caption{Abstand der beiden Tische vom Boden}
    \label{fig:laser-height}
\end{figure}

Die gemessene H\"ohen betrugen nach der Anpassung

\begin{tabular}{ll}
    \hspace{4mm}
    & $h_1 = (\bar{h}_1 \pm s_{\bar{h}_1}) = (256\pm3)\textrm{mm}$ \\
    & $h_2 = (\bar{h}_2 \pm s_{\bar{h}_2}) = (265\pm3)\textrm{mm}$ \\
    & $h_3 = (\bar{h}_3 \pm s_{\bar{h}_3}) = (904\pm1)\textrm{mm}$ \\
    & $h_4 = (\bar{h}_4 \pm s_{\bar{h}_4}) = (895\pm1)\textrm{mm}$ \\
\end{tabular}

In   der   Abbildung   \ref{fig:setup}  ist  eine  Skizze  der  Messeinrichtung.

Der Umlenkspiegel $US$ wird nun in den Laserstrahl platziert und auf den Zentrum
des  Drehspiegels $DS$ ausgerichtet. Durch Feinjustieren des Umlenkspiegels $US$
kann  der  Laserstrahl  genau auf den Zentrum des Drehspiegels $DS$ ausgerichtet
werden.

Der  Motor  des  Drehspiegels  wird  von  Hand  rotiert  bis der Drehspiegel den
Laserstrahl   genau   auf   den   Zentrum   des   Hohlspiegels   $HS$   umlenkt.

Nun  wird der Hohlspiegel $HS$ angepasst und feinjustiert, bis er den Strahl auf
dem Zentrum des Endspiegels $ES$ umlenkt. Dabei ist es wichtig, dass die Distanz
zwischen  den  Spiegeln  $HS$  und  $ES$  genau  $f_2  =  (4985\pm5)\textrm{mm}$
betr\"agt.    Der    Endspiegel    $ES$   wurde   verschoben   und   mit   einem
Laserdistanzmessger\"at  wurde die Distanz auf $(4.988\pm3)\textrm{m}$ gemessen.

Nun wird der Endspiegel $ES$ feinjustiert, bis der Laserstrahl wieder direkt auf
den   Hohlspiegel   $HS$  zur\"uckreflektiert  wird.  Ist  der  Strahl  auf  den
Hohlspiegel   ausgerichtet,   so   schaut   man   auf  den  Drehspiegel  um  den
zur\"uckreflektierenden  Laserstrahl weiter feiner einzustellen. Ist sie auf den
Drehspiegel  genug  genau  ausgerichtet, schaut man weiter auf den Umlenkspiegel
$US$  und  kann somit wieder durch feinjustieren des Endspiegels $ES$ den Strahl
noch weiter genauer ausrichten.

\begin{figure}[H]
    \center
    \includegraphics[width=\textwidth]{images/setup.pdf}
    \caption{Vereinfachte Messeinrichtung mit relevanten Daten}
    \label{fig:setup}
\end{figure}

Nun  wird  der  Strahlteiler  in  den Laserstrahl eingef\"ugt, in der n\"ahe des
Messokulars. Wenn alles stimmt, m\"usste der zur\"uckreflektierender Laserstrahl
nun   durch   das   Messokular   strahlen   und   auf   die   Wand   auftreffen.

Die  Distanz  $S_1$  vom Laserstrahl zum Okular wird gemessen und der Spalt wird
verschoben,  bis  er  auch  mit der Distanz $S_1$ vom Strahlteiler entfernt ist.

Gemessen wurde $S_1 = (107\pm3)\textrm{mm}$.

Jetzt wird die Linse $L_1$ in den Laserstrahl eingef\"ugt. Die Linse muss gleich
der   Brennweite   $f_1   =   1000\textrm{mm}$  vom  Spalt  distanziert  werden.

Gemessen wurde die Distanz von der Linse $L_1$ zum Spalt $(998\pm3)\textrm{mm}$.

Zur \"Ubersicht sind die gemessene Gr\"ossen hier nochmals zusammengefasst:

\begin{tabular}{lp{25mm}l}
    \hspace{4mm}
    & $f_1$ & $(998\pm3)\textrm{mm}$ \\
    & $S_1$ & $(107\pm3)\textrm{mm}$ \\
    & $f_2$ & $(4988\pm5)\textrm{mm}$ \\
    & $S_2$ & $(4894\pm5)\textrm{mm}$ \\
\end{tabular}

Als   letztes   wird   der   Spalt   geschlossen   bis   die   Distanz  $d$  des
Interferenzmusters  (siehe  Abbildung \ref{fig:diffraction}) auf dem Drehspiegel
$DS$    in    etwa    so    breit    ist    wie    der    Drehspiegel    selbst.

\begin{figure}[H]
    \center
    \includegraphics[width=.9\textwidth]{images/diffraction.png}
    \caption{Beugungsmuster des Laserstrahls, verursacht durch den Spalt}
    \label{fig:diffraction}
\end{figure}

Danach  wird  das  Filter  gedreht und der Laserstrahl abgeschw\"acht, damit der
Messvorgang   auch   ohne   Massensterben   von  Netzhautzellen  erfolgen  kann.


\subsection{Versuchsanordnung Kalibrationsmessung}

F\"ur die Kalibrationsmessung wurde wie in  der  Abbildung \ref{fig:kalibration}
vorgegangen.  Ein  Spiegel,  montiert auf einem hochpr\"azisen Arm, wurde in den
Laserstrahl  gestellt.  Der   Arm   wurde   zwischen   $12.5\mu m$   und
$17.5\mu m$  verstellt.   Die   Distanz  wurde  wieder  dabei  gemessen.

\subsection{Messmethoden}

In  der  Tabelle  \ref{tab:messgeraete}  sind  alle  Ger\"ate  aufgelistet,  die
gebraucht  wurden,  um  die Distanzen zwischen den Optikelementen zu messen. Die
Toleranzen  sind grossz\"ugig angegeben.

\begin{center}
    \begin{threeparttable}
        \caption{Liste von Messger\"aten}
        \label{tab:messgeraete}
        \begin{tabular}{ll}
            \toprule
            Bezeichnung                         & Genauigkeit \\
            \midrule
            Disto 08 Laserdistanzmessger\"at    & $\pm1\textrm{mm}$ \\
            $30\textrm{cm}$ Lineal, Metall      & $\pm2\textrm{mm}$ \\
            \bottomrule
        \end{tabular}
    \end{threeparttable}
\end{center}

Es ist zu beachten dass je nach Art der
Messung die Toleranz nochmals zunimmt, wie zum Beispiel beim Messen des Abstands
des Laserstrahls. Die gesch\"atzten Toleranzen sind bei jeder Messung angegeben.


\subsection{Messungen}

Wegen nicht-genaues Lesen der Versuchsanleitung sind uns zwei Details entgangen:
Die  maximale Drehzahl des Motors haben wir bei der Lichtgeschwindigkeitsmessung
\"uberschritten;  es wurde bis 1600 rpm gemessen obwohl der Motor unter 1000 rpm
h\"atte  betrieben werden sollen. Weiter wurde bei der Kalibrationsmessung nicht
im  Bereich von den im Versuchsanleitung angegebenen $\pm1\textrm{mm}$ gemessen,
sondern im Bereich von $\pm2.5\textrm{mm}$.

