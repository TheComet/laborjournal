\section{Fehlerrechnung}

\subsection{Lichtgeschwindigkeitsmessung}

Die  Daten  von  der   Lichtgeschwindigkeitsmessung   wurden   nach  der  Formel
\ref{eq:lichtgeschwindigkeit} gefittet. Es gilt:
\begin{equation}
    x = b \cdot f + x_{02} \hspace{20mm}\textrm{mit}\hspace{20mm} c = \frac{1}{b} \cdot 8\pi \cdot D_1(S_2+D_2)
\end{equation}

Da $b$ f\"ur beide Messungen bekannt ist und die Distanzen zwischen  den  Linsen
gemessen  wurde,  kann die Lichtgeschwindigkeit bestimmt werden. In der  Tabelle
\ref{tab:distanzen}   sind   die  relevanten  Werte  nochmals   zusammengefasst.

\begin{center}
    \begin{threeparttable}
        \caption{Zusammenfassung der relevanten Werte}
        \label{tab:distanzen}
        \begin{tabular}{llp{50mm}}
            \toprule
            $D_1$           & $= \overline{D_1} \pm s_{\overline{D_1}}$                     & $= (998 \pm 3)\textrm{mm}$ \\
            $D_2$           & $= \overline{D_2} \pm s_{\overline{D_2}}$                     & $= (4988 \pm 5)\textrm{mm}$ \\
            $S_2$           & $= \overline{S_2} \pm s_{\overline{S_2}}$                     & $= (4894 \pm 5)\textrm{mm}$ \\
            $b_{Alex}$      & $= \overline{b_{Alex}} \pm s_{\overline{b_{Alex}}}$           & $= -(833.0024 \pm 5.3400) \cdot 10^{-9}$ \\
            $b_{Yohannes}$  & $= \overline{b_{Yohannes}} \pm s_{\overline{b_{Yohannes}}}$   & $= -(839.0909 \pm 8.9991) \cdot 10^{-9}$ \\
            \bottomrule
        \end{tabular}
    \end{threeparttable}
\end{center}

Nun kann die mitlere Lichtgeschwindigkeit berechnet werden. Man beachte dass der
Absolutwert von $b$ f\"ur die  Berechnungen  verwendet  wird,  da  eine negative
Lichtgeschwindigkeit keinen Sinn macht.

\begin{align*}
    \overline{c}            &= \frac{1}{\overline{b}} \cdot 8\pi \cdot \overline{D_1}(\overline{S_2} + \overline{D_2}) \\
    \overline{c_{Alex}}     &= 297.5562 \cdot 10^{6} \textrm{m}/\textrm{s} \\
    \overline{c_{Yohannes}} &= 295.3971 \cdot 10^{6} \textrm{m}/\textrm{s} \\
\end{align*}

Der Fehler wird mit dem \emph{Gauss'schen Fehlerfortpflanzungsgesetz} berechnet.
$c$ ist eine Funktion der gemessenen und gefitteten Werte $b$,  $D_1$, $S_2$ und
$D_2$. Die Partialableitungen der einzelnen Terme sind

\begin{align*}
    \frac{\partial c}{\partial b}   &= -\frac{1}{b^2} \cdot 8\pi \cdot D_1(S_2 + D_2) \\
    \frac{\partial c}{\partial D_1} &= \frac{1}{b} \cdot 8\pi \cdot (S_2 + D_2) \\
    \frac{\partial c}{\partial D_2} &= \frac{1}{b} \cdot 8\pi \cdot D_1(S_2 + 1) \\
    \frac{\partial c}{\partial S_2} &= \frac{1}{b} \cdot 8\pi \cdot D_1(1 + D_2) \\
\end{align*}

und   somit    ergeben    sich    die   Fehler   $s_{\overline{c_{Alex}}}$   und
$s_{\overline{c_{Yohannes}}}$ als

\begin{align*}
    s_{\overline{c}} &= \sqrt{ \left( \left.\frac{\partial \overline{c}}{\partial \overline{b  }} \right\rvert_{\overline{c}} \cdot s_{\overline{b  }} \right)^2
                             + \left( \left.\frac{\partial \overline{c}}{\partial \overline{D_1}} \right\rvert_{\overline{c}} \cdot s_{\overline{D_1}} \right)^2
                             + \left( \left.\frac{\partial \overline{c}}{\partial \overline{D_2}} \right\rvert_{\overline{c}} \cdot s_{\overline{D_2}} \right)^2
                             + \left( \left.\frac{\partial \overline{c}}{\partial \overline{S_2}} \right\rvert_{\overline{c}} \cdot s_{\overline{S_2}} \right)^2 } \\
    s_{\overline{c_{Alex}}}     &= 2.46 \cdot 10^6 \textrm{m}/\textrm{s} \\
    s_{\overline{c_{Yohannes}}} &= 3.52 \cdot 10^6 \textrm{m}/\textrm{s} \\
\end{align*}

Die  zwei  unabh\"angig gemessene Lichtgeschwindigkeiten k\"onnen nun  gewichtet
gemittelt werden um an einen genaureren Wert zu gelangen. Es gilt:
\begin{align*}
    \overline{x}  &= \frac{\sum_{i=1}^{N} g_{\overline{x}_i} \cdot \overline{x}_i}{\sum_{i=1}^{N} g_{\overline{x}_i}} \hspace{8mm}\textrm{mit}\hspace{8mm} g_{\overline{x}_i} = \frac{1}{s^2_{\overline{x}_i}} \\
\end{align*}
und somit
\begin{align*}
    \overline{c}  &= \frac{ \frac{1}{s_{\overline{c_{Alex}}}^2} \cdot \overline{c_{Alex}} + \frac{1}{s_{\overline{c_{Yohannes}}}^2} \cdot \overline{c_{Yohannes}} }
                          { \frac{1}{s_{\overline{c_{Alex}}}^2} + \frac{1}{s_{\overline{c_{Yohannes}}}^2} } \\
                  &= 296.85 \cdot 10^6 \textrm{m}/\textrm{s} \\
\end{align*}

F\"ur den Fehler des gewogenen Mittels gilt:
\begin{align*}
    s_{\overline{x}} &= \frac{1}{\sqrt{\sum_{i=1}^{N} g_{\overline{x}_i}}}
\end{align*}
und somit
\begin{align*}
    s_{\overline{c}} &= \frac{1}{ \sqrt{\frac{1}{s_{\overline{c_{Alex}}}^2} + \frac{1}{s_{\overline{c_{Yohannes}}}^2}}} \\
                     &= 2.02 \cdot 10^6 \textrm{m}/\textrm{s} \\
\end{align*}

Die Lichtgeschwindigkeit \textbf{ohne Kalibration} ergibt als:
\begin{align*}
    c = \overline{c} + s_{\overline{c}} = \underline{\underline{(296.8 \pm 2.0) \cdot 10^6 \textrm{m}/\textrm{s}}}
\end{align*}


\subsection{Kalibrationsmessung}

Da  $a$  und  $b$ bekannt sind und die Distanzen zwischen  den  Linsen  gemessen
wurde,  kann  die genauere Lichtgeschwindigkeit bestimmt werden. In der  Tabelle
\ref{tab:kalibration-werte} sind die relevanten Werte nochmals  zusammengefasst.

\begin{center}
    \begin{threeparttable}
        \caption{Zusammenfassung der relevanten Werte}
        \label{tab:kalibration-werte}
        \begin{tabular}{llp{50mm}}
            \toprule
            $D_2$           & $= \overline{D_2} \pm s_{\overline{D_2}}$                     & $= (4988 \pm 5)\textrm{mm}$ \\
            $S_2$           & $= \overline{S_2} \pm s_{\overline{S_2}}$                     & $= (4894 \pm 5)\textrm{mm}$ \\
            $a_{Alex}$      & $= \overline{a_{Alex}} \pm s_{\overline{a_{Alex}}}$           & $= (1999.827 \pm 2.461) \cdot 10^-3$ \\
            $a_{Yohannes}$  & $= \overline{a_{Yohannes}} \pm s_{\overline{a_{Yohannes}}}$   & $= (2002.809 \pm 3.498) \cdot 10^-3$ \\
            $b_{Alex}$      & $= \overline{b_{Alex}} \pm s_{\overline{b_{Alex}}}$           & $= -(833.0024 \pm 5.3400) \cdot 10^{-9}$ \\
            $b_{Yohannes}$  & $= \overline{b_{Yohannes}} \pm s_{\overline{b_{Yohannes}}}$   & $= -(839.0909 \pm 8.9991) \cdot 10^{-9}$ \\
            \bottomrule
        \end{tabular}
    \end{threeparttable}
\end{center}

Die   Daten   von   der    Kalibrationsmessung    wurden    nach    der   Formel
\ref{eq:kalibration-a}  gefittet.   Die   daraus  gewonnene  Steigung  $a$  kann
verwendet   werden  um  eine  genauere   Berechnung   der   Lichtgeschwindigkeit
durchzuf\"uhren. Es gilt dabei die Formel \ref{eq:lichtgeschwindigkeit_genauer}.
\begin{align*}
\end{align*}

Die mittlere Lichtgeschwindigkeiten berechnen sich wie folgt.
\begin{align*}
    \overline{c}            &= 4\pi\frac{\overline{a}}{\overline{b}}(S_2 + D_2) \\
    \overline{c_{Alex}}     &= 298.13 \cdot 10^6 \textrm{m}/\textrm{s} \\
    \overline{c_{Yohannes}} &= 296.40 \cdot 10^6 \textrm{m}/\textrm{s} \\
\end{align*}

Der Fehler wird mit dem \emph{Gauss'schen Fehlerfortpflanzungsgesetz} berechnet.
$c$ ist eine Funktion  der  gemessenen  und gefitteten Werte $a$, $b$, $D_2$ und
$S_2$. Die Partialableitungen der einzelnen Terme sind
\begin{align*}
    \frac{\partial c}{\partial a}   &= 4\pi\frac{1}{b}(S_2+D_2) \\
    \frac{\partial c}{\partial b}   &= -4\pi\frac{a}{b^2}(S_2+D_2) \\
    \frac{\partial c}{\partial S_2} &= 4\pi\frac{a}{b}(1+D_2) \\
    \frac{\partial c}{\partial D_2} &= 4\pi\frac{a}{b}(S_2+1) \\
\end{align*}

und somit ergeben sich die Fehler $s_{\overline{c_{Alex}}}$ und $s_{\overline{c_{Yohannes}}}$ als
\begin{align*}
    s_{\overline{c}}            &= \sqrt{ \left( \left.\frac{\partial \overline{c}}{\partial \overline{a  }} \right\rvert_{\overline{c}} \cdot s_{\overline{a  }} \right)^2
                                        + \left( \left.\frac{\partial \overline{c}}{\partial \overline{b  }} \right\rvert_{\overline{c}} \cdot s_{\overline{b  }} \right)^2
                                        + \left( \left.\frac{\partial \overline{c}}{\partial \overline{S_2}} \right\rvert_{\overline{c}} \cdot s_{\overline{S_2}} \right)^2
                                        + \left( \left.\frac{\partial \overline{c}}{\partial \overline{D_2}} \right\rvert_{\overline{c}} \cdot s_{\overline{D_2}} \right)^2 } \\
    s_{\overline{c_{Alex}}}     &= 2.32 \cdot 10^6 \textrm{m}/\textrm{s} \\
    s_{\overline{c_{Yohannes}}} &= 3.46 \cdot 10^6 \textrm{m}/\textrm{s} \\
\end{align*}

Die zwei unabh\"angig gemessene Lichtgeschwindigkeiten  k\"onnen  nun  gewichtet
gemittelt  werden   um   an   einen   genaueren   Wert  zu  gelangen.  Es  gilt:
\begin{align*}
    \overline{x} &= \frac{\sum_{i=1}^{N} g_{\overline{x}_i} \cdot \overline{x}_i}{\sum_{i=1}^{N} g_{\overline{x}_i}} \hspace{8mm}\textrm{mit}\hspace{8mm} g_{\overline{x}_i} = \frac{1}{s^2_{\overline{x}_i}} \\
\end{align*}
und somit
\begin{align*}
    \overline{c}  &= \frac{ \frac{1}{s_{\overline{c_{Alex}}}^2} \cdot \overline{c_{Alex}} + \frac{1}{s_{\overline{c_{Yohannes}}}^2} \cdot \overline{c_{Yohannes}} }
                          { \frac{1}{s_{\overline{c_{Alex}}}^2} + \frac{1}{s_{\overline{c_{Yohannes}}}^2} } \\
                  &= 297.59 \cdot 10^6 \textrm{m}/\textrm{s} \\ 
\end{align*}

F\"ur den Fehler des gewogenen Mittels gilt:
\begin{align*}
    s_{\overline{x}} &= \frac{1}{\sqrt{\sum_{i=1}^{N} g_{\overline{x}_i}}}
\end{align*}
und somit
\begin{align*}
    s_{\overline{c}} &= \frac{1}{ \sqrt{\frac{1}{s_{\overline{c_{Alex}}}^2} + \frac{1}{s_{\overline{c_{Yohannes}}}^2}}} \\
                     &= 1.93 \cdot 10^6 \textrm{m}/\textrm{s} \\
\end{align*}

Die Lichtgeschwindigkeit \textbf{mit Kalibration} ergibt als:
\begin{align*}
    c = \overline{c} + s_{\overline{c}} = \underline{\underline{(297.6 \pm 1.9) \cdot 10^6 \textrm{m}/\textrm{s}}}
\end{align*}

