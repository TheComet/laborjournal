\section{Fehlerrechnung}

\subsection{Lichtgeschwindigkeitsmessung}

Die  Daten  von  der   Lichtgeschwindigkeitsmessung   wurden   nach  der  Formel
\ref{eq:lichtgeschwindigkeit} gefittet. Es gilt:
\begin{equation}
    x = b_1 \cdot f + a_1 \hspace{15mm}\textrm{mit}\hspace{15mm} c = \frac{1}{b_1} \cdot 8\pi \cdot D_1(S_2+D_2)
\end{equation}

Da $b$ f\"ur beide Messungen bekannt ist und die Distanzen zwischen  den  Linsen
gemessen  wurde,  kann die Lichtgeschwindigkeit bestimmt werden. In der  Tabelle
\ref{tab:distanzen}   sind   die  relevanten  Werte  nochmals   zusammengefasst.

\begin{center}
    \begin{threeparttable}
        \caption{Zusammenfassung der relevanten Werte}
        \label{tab:distanzen}
        \begin{tabular}{llp{50mm}}
            \toprule
            $D_1$           & $= \overline{D_1} \pm s_{\overline{D_1}}$                     & $= (1.004 \pm 3)\textrm{mm}$ \\
            $D_2$           & $= \overline{D_2} \pm s_{\overline{D_2}}$                     & $= (5000 \pm 3)\textrm{mm}$ \\
            $S_2$           & $= \overline{S_2} \pm s_{\overline{S_2}}$                     & $= (4998 \pm 3)\textrm{mm}$ \\
            $b_{1,Remo}$    & $= \overline{b_{1,Remo}} \pm s_{\overline{b_{1,Remo}}}$       & $= (856.59 \pm 29.96)\cdot 10^{-9}$ \\
            $b_{1,Alex}$    & $= \overline{b_{1,Alex}} \pm s_{\overline{b_{1,Alex}}}$       & $= (892.12 \pm 20.68)\cdot 10^{-9}$ \\
            \bottomrule
        \end{tabular}
    \end{threeparttable}
\end{center}

Nun kann die mitlere Lichtgeschwindigkeit berechnet  werden.

\begin{align*}
    \overline{c}            &= \frac{1}{\overline{b_1}} \cdot 8\pi \cdot \overline{D_1}(\overline{S_2} + \overline{D_2}) \\
    \overline{c_{Remo}}     &= 294.519 \cdot 10^{6} \textrm{m}/\textrm{s} \\
    \overline{c_{Alex}}     &= 282.790 \cdot 10^{6} \textrm{m}/\textrm{s} \\
\end{align*}

Der  Fehler  wird   mit   dem   \emph{Gauss'schen  Fehlerfortpflanzungsgesetz}
berechnet. $c$ ist eine Funktion der gemessenen  und  gefitteten  Werte $b_1$,
$D_1$,  $S_2$ und $D_2$.  Die  Partialableitungen  der  einzelnen  Terme  sind

\begin{align*}
    \frac{\partial c}{\partial b_1} &= -\frac{1}{b_1^2} \cdot 8\pi \cdot D_1(S_2 + D_2) \\
    \frac{\partial c}{\partial D_1} &= \frac{1}{b_1} \cdot 8\pi \cdot (S_2 + D_2) \\
    \frac{\partial c}{\partial D_2} &= \frac{1}{b_1} \cdot 8\pi \cdot D_1(S_2 + 1) \\
    \frac{\partial c}{\partial S_2} &= \frac{1}{b_1} \cdot 8\pi \cdot D_1(1 + D_2) \\
\end{align*}

und   somit    ergeben    sich    die   Fehler   $s_{\overline{c_{Remo}}}$   und
$s_{\overline{c_{Alex}}}$ als

\begin{align*}
    s_{\overline{c}} &= \sqrt{ \left( \left.\frac{\partial \overline{c}}{\partial \overline{b_1}} \right\rvert_{\overline{c}} \cdot s_{\overline{b_1}} \right)^2
                             + \left( \left.\frac{\partial \overline{c}}{\partial \overline{D_1}} \right\rvert_{\overline{c}} \cdot s_{\overline{D_1}} \right)^2
                             + \left( \left.\frac{\partial \overline{c}}{\partial \overline{D_2}} \right\rvert_{\overline{c}} \cdot s_{\overline{D_2}} \right)^2
                             + \left( \left.\frac{\partial \overline{c}}{\partial \overline{S_2}} \right\rvert_{\overline{c}} \cdot s_{\overline{S_2}} \right)^2 } \\
    s_{\overline{c_{Remo}}}     &= 10.37 \cdot 10^6 \textrm{m}/\textrm{s} \\
    s_{\overline{c_{Alex}}} &= 6.65 \cdot 10^6 \textrm{m}/\textrm{s} \\
\end{align*}

Die  zwei  unabh\"angig gemessene Lichtgeschwindigkeiten k\"onnen nun  gewichtet
gemittelt werden um an einen genaureren Wert zu gelangen. Es gilt:
\begin{align*}
    \overline{x}  &= \frac{\sum_{i=1}^{N} g_{\overline{x}_i} \cdot \overline{x}_i}{\sum_{i=1}^{N} g_{\overline{x}_i}} \hspace{8mm}\textrm{mit}\hspace{8mm} g_{\overline{x}_i} = \frac{1}{s^2_{\overline{x}_i}} \\
\end{align*}
und somit
\begin{align*}
    \overline{c}  &= \frac{ \frac{1}{s_{\overline{c_{Remo}}}^2} \cdot \overline{c_{Remo}} + \frac{1}{s_{\overline{c_{Alex}}}^2} \cdot \overline{c_{Alex}} }
                          { \frac{1}{s_{\overline{c_{Remo}}}^2} + \frac{1}{s_{\overline{c_{Alex}}}^2} } \\
                  &= 286.21 \cdot 10^6 \textrm{m}/\textrm{s} \\
\end{align*}

F\"ur den Fehler des gewogenen Mittels gilt:
\begin{align*}
    s_{\overline{x}} &= \frac{1}{\sqrt{\sum_{i=1}^{N} g_{\overline{x}_i}}}
\end{align*}
und somit
\begin{align*}
    s_{\overline{c}} &= \frac{1}{ \sqrt{\frac{1}{s_{\overline{c_{Remo}}}^2} + \frac{1}{s_{\overline{c_{Alex}}}^2}}} \\
                     &= 5.60 \cdot 10^6 \textrm{m}/\textrm{s} \\
\end{align*}

Die Lichtgeschwindigkeit \textbf{ohne Kalibration} ergibt als:
\begin{align*}
    c = \overline{c} + s_{\overline{c}} = \underline{\underline{(286.21 \pm 5.60) \cdot 10^6 \textrm{m}/\textrm{s}}}
\end{align*}


\subsection{Kalibrationsmessung}

Da $b_1$ und $b_2$ bekannt sind und die Distanzen zwischen den Linsen gemessen
wurde, kann die genauere Lichtgeschwindigkeit bestimmt werden. In  der Tabelle
\ref{tab:kalibration-werte}    sind    die     relevanten    Werte    nochmals
zusammengefasst.

\begin{center}
    \begin{threeparttable}
        \caption{Zusammenfassung der relevanten Werte}
        \label{tab:kalibration-werte}
        \begin{tabular}{llp{50mm}}
            \toprule
            $D_2$            & $= \overline{D_2} \pm s_{\overline{D_2}}$                     & $= (5000 \pm 3)\textrm{mm}$ \\
            $S_2$            & $= \overline{S_2} \pm s_{\overline{S_2}}$                     & $= (4998 \pm 3)\textrm{mm}$ \\
            $b_{1,Remo}$     & $= \overline{b_{1,Remo}} \pm s_{\overline{b_{1,Remo}}}$       & $= (856.59 \pm 29.96) \cdot 10^{-9}$ \\
            $b_{1,Alex}$     & $= \overline{b_{1,Alex}} \pm s_{\overline{b_{1,Alex}}}$       & $= (892.12 \pm 20.68) \cdot 10^{-9}$ \\
            $b_{2,Remo}$     & $= \overline{b_{2,Remo}} \pm s_{\overline{b_{2,Remo}}}$       & $= (2030.5 \pm 7-0) \cdot 10^{-3}$ \\
            $b_{2,Alex}$     & $= \overline{b_{2,Alex}} \pm s_{\overline{b_{2,Alex}}}$       & $= (2018.2 \pm 8.9) \cdot 10^{-3}$ \\
            \bottomrule
        \end{tabular}
    \end{threeparttable}
\end{center}

Die   Daten   von   der    Kalibrationsmessung    wurden   nach   der   Formel
\ref{eq:kalibration-a} gefittet. Die  daraus  gewonnene  Steigung  $b_2$  kann
verwendet  werden  um  eine  genauere  Berechnung   der   Lichtgeschwindigkeit
durchzuf\"uhren.        Es        gilt        dabei         die         Formel
\ref{eq:lichtgeschwindigkeit_genauer}.

Die mittlere Lichtgeschwindigkeiten berechnen sich wie folgt.

\begin{align*}
    \overline{c}            &= 4\pi\frac{\overline{b_2}}{\overline{b_1}}(S_2 + D_2) \\
    \overline{c_{Remo}}     &= 297.819 \cdot 10^6 \textrm{m}/\textrm{s} \\
    \overline{c_{Alex}}     &= 284.226 \cdot 10^6 \textrm{m}/\textrm{s} \\
\end{align*}

Der Fehler wird mit dem \emph{Gauss'schen Fehlerfortpflanzungsgesetz} berechnet.
$c$ ist eine Funktion  der  gemessenen  und gefitteten Werte $b_1$, $b_2$, $D_2$ und
$S_2$. Die Partialableitungen der einzelnen Terme sind

\begin{align*}
    \frac{\partial c}{\partial b_2}   &= 4\pi\frac{1}{b_1}(S_2+D_2) \\
    \frac{\partial c}{\partial b_1}   &= -4\pi\frac{b_2}{b_1^2}(S_2+D_2) \\
    \frac{\partial c}{\partial S_2}   &= 4\pi\frac{b_2}{b_1}(1+D_2) \\
    \frac{\partial c}{\partial D_2}   &= 4\pi\frac{b_2}{b_1}(S_2+1) \\
\end{align*}

und    somit   ergeben   sich   die   Fehler   $s_{\overline{c_{Remo}}}$   und
$s_{\overline{c_{Alex}}}$ als

\begin{align*}
    s_{\overline{c}}            &= \sqrt{ \left( \left.\frac{\partial \overline{c}}{\partial \overline{b_2}} \right\rvert_{\overline{c}} \cdot s_{\overline{b_2}} \right)^2
                                        + \left( \left.\frac{\partial \overline{c}}{\partial \overline{b_1}} \right\rvert_{\overline{c}} \cdot s_{\overline{b_1}} \right)^2
                                        + \left( \left.\frac{\partial \overline{c}}{\partial \overline{S_2}} \right\rvert_{\overline{c}} \cdot s_{\overline{S_2}} \right)^2
                                        + \left( \left.\frac{\partial \overline{c}}{\partial \overline{D_2}} \right\rvert_{\overline{c}} \cdot s_{\overline{D_2}} \right)^2 } \\
    s_{\overline{c_{Remo}}}     &= 10.49 \cdot 10^6 \textrm{m}/\textrm{s} \\
    s_{\overline{c_{Alex}}}     &= 6.75 \cdot 10^6 \textrm{m}/\textrm{s} \\
\end{align*}

Die zwei unabh\"angig gemessene Lichtgeschwindigkeiten  k\"onnen  nun  gewichtet
gemittelt  werden   um   an   einen   genaueren   Wert  zu  gelangen.  Es  gilt:

\begin{align*}
    \overline{x} &= \frac{\sum_{i=1}^{N} g_{\overline{x}_i} \cdot \overline{x}_i}{\sum_{i=1}^{N} g_{\overline{x}_i}} \hspace{8mm}\textrm{mit}\hspace{8mm} g_{\overline{x}_i} = \frac{1}{s^2_{\overline{x}_i}} \\
\end{align*}

und somit

\begin{align*}
    \overline{c}  &= \frac{ \frac{1}{s_{\overline{c_{Remo}}}^2} \cdot \overline{c_{Remo}} + \frac{1}{s_{\overline{c_{Alex}}}^2} \cdot \overline{c_{Alex}} }
                          { \frac{1}{s_{\overline{c_{Remo}}}^2} + \frac{1}{s_{\overline{c_{Alex}}}^2} } \\
                  &= 288.20 \cdot 10^6 \textrm{m}/\textrm{s} \\ 
\end{align*}

F\"ur den Fehler des gewogenen Mittels gilt:

\begin{align*}
    s_{\overline{x}} &= \frac{1}{\sqrt{\sum_{i=1}^{N} g_{\overline{x}_i}}}
\end{align*}

und somit

\begin{align*}
    s_{\overline{c}} &= \frac{1}{ \sqrt{\frac{1}{s_{\overline{c_{Remo}}}^2} + \frac{1}{s_{\overline{c_{Alex}}}^2}}} \\
                     &= 5.67 \cdot 10^6 \textrm{m}/\textrm{s} \\
\end{align*}

Die Lichtgeschwindigkeit \textbf{mit Kalibration} ergibt als:

\begin{align*}
    c = \overline{c} + s_{\overline{c}} = \underline{\underline{(288.20 \pm 5.67) \cdot 10^6 \textrm{m}/\textrm{s}}}
\end{align*}

