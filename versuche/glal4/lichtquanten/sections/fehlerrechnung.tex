\section{Fehlerrechnung}

Es  wurden  auf drei verschiedene Arten die Steigung  $m_{ph},  m_0  und  m_k$
bestummen.  Die  Plank'sche  Konstante  $h$  kann mit  der  Formel  \ref{eq:h}
berechnet werden.

Die Formel \ref{eq:h} wird mit in  das  Gaussischen Fehlerfortpflanzungsgesetz
eingesetzt, um den Fehler zu Berechnen:

\begin{align*}
    h &= m \cdot e \hspace{5mm} \frac{\partial h}{\partial m} = e \hspace{5mm} \frac{\partial h}{\partial e} = m \\
    s_h &= \sqrt{\left(\overline{e} \cdot s_e\right)^2 + \left(\overline{m} \cdot s_m\right)^2}
\end{align*}

Da  wir  die Elementarladung $e$ so viel genauer bekannt ist als die gemessene
Steigung,  kann   $e_s   \approx  0$  angenommen  werden.  Die  Fehlerrechnung
vereinfacht sich auf:

\begin{equation*}
    s_h = \overline{m} \cdot s_m
\end{equation*}

Das Experiment  ist  verschiedenen  Einfl\"ussen unterworfen, welche nicht nur
die  Messgenauigkeit beeintr\"achtigen, sondern auch zu systematischen Fehlern
f\"uhren.  Zun\"achst  ist  die  Erfassung  der  kritischen  Gegenspannung  Uo
unsicher, weil das Verschwinden  bzw. Einsetzen des Photostromes "schleichend"
ist,  bedingt  durch  das  breite  Energiespektrum  der  Photoelektronen. Dann
f\"uhren   Inhomogenit\"aten  und  Alterungseffekte  (z.B.Gasadsorption)   der
Kathodenoberfl\"ache dazu, dass die Austrittsarbeit  Wa \"ortlich und zeitlich
variert.

Es ist mit  einer  Systematischen  Abweichung  von  etwa \SI{10}{\percent} bis
\SI{30}{\percent} rechnen. Die  statistischen  Fehler  werden  also angepasst:

\begin{align*}
    s_{h_{ph}} &:= \sqrt{s_{h_{ph}}^2 + 0.3 \cdot h_{ph}} \\
    s_{h_0}    &:= \sqrt{s_{h_0}^2 + 0.3 \cdot h_0} \\
    s_{h_k}    &:= \sqrt{s_{h_k}^2 + 0.3 \cdot h_k}
\end{align*}

\begin{align*}
    h_{ph} = \overline{h_{ph}} \pm s_{h_{ph}} &= m_{ph} \cdot e \pm e \cdot s_{m_{ph}} = \left(5.49771 \pm 1.65266\right)e-34 \\
    h_0    = \overline{h_0}    \pm s_{h_0}    &= m_0    \cdot e \pm e \cdot s_{m_0}    = \left(5.24227 \pm 1.58724\right)e-34 \\
    h_k    = \overline{h_k}    \pm s_{h_k}    &= m_k    \cdot e \pm e \cdot s_{m_k}    = \left(7.54210 \pm 2.55475\right)e-34
\end{align*}

Die   drei  Resultate  k\"onnen  jetzt   gewichtet   gemittelt   werden,   mit
Fehlerrechnung:

\begin{equation*}
    h = \overline{h} \pm s_h = \frac{\sum_i g_{h_i}\cdot h_i}{\sum_i g_{h_i}} \pm \frac{1}{\sqrt{\sum_i g_{h_i}}}
    \hspace{5mm}\textrm{mit den Gewichten}\hspace{5mm}
    g_{h_i} = \frac{1}{s_{h_i}^2}
\end{equation*}

Es ergibt sich die Plank'sche Konstante:

\begin{equation*}
    h = \overline{h} \pm s_h = \left(5.72891 \pm 1.04469\right)e-34
\end{equation*}

