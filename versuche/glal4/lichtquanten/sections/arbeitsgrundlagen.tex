\section{Arbeitsgrundlagen}

Die Ausbreitung von Licht und deren Wechselwirkung mit  materiellen  K\"orpern
kann in zwei Bereichen aufgeteilt werden.


\subsection{Wellentheorie der Elektromagnetischen Strahlung}

Licht wird als elektromagnetische  Welle  modelliert,  darstellbar  durch eine
kontinuierliche Wellenfunktion,  zum  Beispiel  $\vec{E}(\vec{r},t)$ f\"ur die
elektrische Feldst\"arke.

Zu  dieser  Aufgabe  relevante  charakteristische  Gr\"ossen  sind  dabei  die
Wellenl\"ange  $\lambda$  und  die  Frequenz  $f$, wobei sich  die  Welle  mit
Lichtgeschwindigkeit $c$ ausbreitet.

\begin{equation}
    \lambda \cdot f = c
\end{equation}


\subsection{Quantentheorie der elektromagnetischen Strahlung}

Licht  wird  als  ein  Strom  von  Photonen $\gamma$ interpretiert, welche  in
\textbf{individuellen Quantenprozessen} mit  einzelnen  Atomen, Elektronen und
so weiter  in  wechselwirkung treten. Diese unterliegt dem Zufall, weshalb nur
\textbf{Wahrscheinlichkeitsaussagen} gemacht werden k\"onnen.

Zu dieser Aufgabe relevante charakteristische Gr\"ossen sind dabei die Energie
$E_{\gamma}$ und die Plank'sche Konstante  $h$  (welches wir in diesem Versuch
bestimmen werden).

\begin{equation}
    E_{\gamma} = h \cdot f = \frac{h \cdot c}{\lambda}
\end{equation}


\subsection{Der Photoeffekt}

Beim Photoeffekt  handelt  es  sich  um  die Absorption von Lichtquanten durch
Atome, Mole\"ule, Festk\"orper, bei der  das  Photon verschwindet und ein Teil
seiner  Energie  auf  ein  Elektron  \"ubertragen  wird.  Im Falle von Atomen,
Mole\"ulen  enstehen  im  ersten Momen freie Elektronen mit kinetische Energie
$E_k = E_{\gamma} -  E_B$  (Die  am  Prozess  beteiligten  Atome  sind so viel
schwerer, dass ihre Energie vernachl\"assigt werden kann), welche sich dann an
andere  Teilchen  anlagern   k\"onnen.   $E_B$  ist  die  Bindungsenergie  des
Elektrons.  Im Falle der Wechselwirkung mit Festk\"orpern wird  unterschieden:


\subsubsection{\"Ausserer Photoeffekt}

Die Lichtquanten ``schlagen'' aus der Oberfl\"ache von  Metallen, Metalloxiden
und  Halbleitern  Photoelektronen  heraus. Zur Befreiung eines Elektrons  muss
dessen  Bindungsenergie  aufgewendet  werden,  man  bezeichnet  sie  hier  als
Austrittsarbeit  $W_a$.  Es ist klar, dass  die  Photonenenergie  $E_{\gamma}$
mindestens  gleich  $W_a$  sein muss, damit der Effekt eintritt.  Der  Prozess
spielt   sich   in   einer   d\"unnen  Oberfl\"achenschicht  ab,  welche   der
Eindringtiefe der Photonen entspricht1) . Deshalb erleiden die Photoelektronen
vor dem Austritt durch Wechselwirkung  mit dem Kristallgitter unterschiedliche
Energieverluste und verlassen die Oberfl\"ache mit variabler  kinet.  Energie.
Der Maximalwert ist gegeben durch die Gleichung:

\begin{equation}
    E_k = E_{\gamma} - W_a = h \cdot f - W_a
\end{equation}

welche erstmals von  \textit{Einstein}  formuliert  wurde. Die Austrittsarbeit
ist   eine   Oberfl\"acheneigenschaft   und   nur   bei   gr\"osster  Reinheit
materialspezifisch.   Adsorbierte  Gase,  Oxidfilme  und  andere   Fremdstoffe
k\"onnen den  Wert erheblich ver\"andern. Schichten, welche zur Umwandlung von
Strahlung  in   einen   Photoelektronenstrom   dienen,   bezeichnet   man  als
Photokathoden; wird eine solche  mit  einem  Sekund\"arelektronenvervielfacher
(engl. \textit{Multiplier}) kombiniert,  erh\"alt  man  einen Photomultiplier,
den empfindlichsten Strahlungsdetektor.


\subsubsection{Innerer Photoeffekt}

Die in einen  Halbleiter  eindringenden Lichtquanten produzieren zus\"atzliche
Ladungstr\"ager, wir unterscheiden:

Photoleiter     (Photowiderst\"ande):     bei    Bestrahlung     nimmt     die
Ladungstr\"agerdichte und damit die Leitf\"ahigkeit zu. Beispiele sind: CdS im
Sichtbaren, InAs und InSb im nahen IR, dotiertes Ge bis ins  ferne  IR.  Wegen
ihrer  Einfachheit  werden  solche  Elemente  f\"ur  zahlreiche   Messaufgaben
eingesetzt.   Photodioden:  im  Bereich  des  p-n-\"Uberganges  erzeugen   die
absorbierten Photonen Elektron-Loch- Paare, welche im  elektrischen  Feld  der
Raumladungszone  getrennt  werden. Ohne  \"aussere  Quelle  wird  die  p-Seite
positiv und  die  n-Seite  negativ aufgeladen, es entsteht eine Photospannung,
die  maximal entgegengesetzt gleich der Diffusionsspannung $U_d$ werden  kann.
Bei Belastung  fliesst ein entsprechender Strom ( Elementbetrieb) zum Beispiel
bei Solarzellen. Wird eine \"aussere Spannung  in  Sperrrichtung  angelegt, so
fliesst  in  dieser  Richtung  ein Photostrom,  der  streng  proportional  zur
auftreffenden  Strahlungsleistung   ist  (  Diodenbetrieb),  angewendet  f\"ur
Photodetektoren.  Diese   Vorg\"ange  laufen  nat\"urlich  nur  ab,  wenn  die
Photonenenergie  einen  materialspezifischen  Schwellenwert   \"uberschreitet,
welcher  der  Freisetzung  eines  ans   Kristallgitter   gebundenen  Elektrons
entspricht.  In  eigenleitendem Material ist der energetische Abstand zwischen
Leitungs-  und  Valenzband = Breite der Bandl\"ucke (engl. \textit{gap}) $E_g$
massgebend.  Dies  ergibt  die  langwellige Grenze  der  Photoempfindlichkeit.


\subsection{Bestimmung der Plank'schen Konstante h}

Elektronen  in  der  Photokathode  absorbieren  die  Energie  $h\cdot\nu$  der
einfallenden Photonen. Ist  ihre  Energie anschließend hoch genug, so k\"onnen
sie  das  Metall  verlassen  und  fuhren so zu einer positiven  Aufladung  der
beleuchteten Elektrode. Die Elektronen haben nach  dem  Verlassen  der Kathode
eine gewisse  maximale  kinetische  Energie $E_{kin}$, die aber nicht ganz der
Energie  der absorbierten Photonen entspricht, sondern  etwas  niedriger  ist.
Dies  liegt  daran,  dass  beim  Austritt  aus  jedem  Material die sogenannte
Austrittsarbeit $W_A$  aufgebracht werden muss, bzw. beim Eintritt wieder frei
wird. Es gilt die Gleichung:

\begin{equation}
    E_{kin} = \frac{1}{2}m_e v^2 = h\cdot\nu - W_{A,Kathode}
\end{equation}

Die  Messung  der  maximalen  kinetischen  Energie  der  Elektronen  kann  auf
verschiedene    Arten   erfolgen.   Denkbar   w\"are    zum    Beispiel    die
Fluggeschwindigkeit  direkt  zu   bestimmen.  Allerdings  w\"are  ein  daf\"ur
geeigneter  Aufbau  sehr aufw\"andig. Einfacher ist es, zwischen  Kathode  und
Anode eine  Gegenspannung anzulegen, die gerade so eingestellt wird, dass auch
die  schnellsten  Elektronen  die  Anode  gerade  nicht  mehr  erreichen.  Der
(relativ) einfach messbare  Photostrom  wird  fur  diesen  Wert $U_{max}$ dann
gerade null. $U_{max}\cdot e$ entspricht bis  auf  einen  konstanten Summanden
der kinetischen Energie $E_{kin}$. Die Konstante kommt daher, dass nat\"urlich
nicht nur die Kathode, sondern auch die Anode eine Austrittsarbeit hat. Da die
beiden Elektroden \"uber das Messger\"at  verbunden  sind,  muss auch noch die
Kontaktspannung    berucksichtigt   werden,   so   dass   schließlich    gilt:

\begin{align}
    U_{max}(\nu)\cdot e &= E_{kin} - U_{Kontakt} \cdot e \\
                        &= \left(h\cdot\nu - W_{A,Kathode}\right) - \left(W_{A,Anode}-W_{A,Kathode}\right) \\
                        &= h\cdot\nu - W_{A,Anode}
    \label{eq:umax}
\end{align}

Die Austrittsarbeit der Kathode spielt f\"ur den Wert  der  Gegenspannung also
tats\"achlich keine Rolle. Ihre Gr\"osse entscheidet aber nat\"urlich trotzdem
dar\"uber,  ob  \"uberhaupt  Photoelektronen emittiert werden k\"onnen und wie
hoch  deren kinetische Energie ist. Die Austrittsarbeiten h\"angen  nicht  von
der  Frequenz  des  eingestrahlten Lichtes ab. Deshalb kann man auch ohne ihre
Kenntnis  mit  Hilfe  der  Gleichung  \ref{eq:umax}  aus  zwei  Messungen  von
$U_max(\nu)$  f\"ur  verschiedene  Lichtfrequenzen  $\nu_1$  und $\nu_2$ durch
Differenzbildung die Konstante $h$ bestimmen:

\begin{equation}
    h = \frac{e\cdot\left(U_{max}(\nu_2) - U_{max}(\nu_1)\right)}{\nu_2 - \nu_1}
\end{equation}

Oder, falls mehrere Messdaten vorhanden sind, kann mittels linearer Regression
die Konstante $h$ auch anhand der Steigung berechnet werden:

\begin{equation}
    h = e \cdot m
    \label{eq:h}
\end{equation}

Wobei $m$ die Steigung des Fits der Funktion  $U_{max}(\nu)  =  m\cdot\nu + q$
ist.

Die  Tatsache,  dass  zwar  die  Anzahl, jedoch (zumindest fur  konventionelle
Lichtquellen)   nicht   die   Energie  der  austretenden  Elektronen  von  der
Intensit\"at  der   Strahlung   abh\"angt,   f\"uhrte   um   1900  zu  grossen
Erkl\"arungsschwierigkeiten im Rahmen der allgemein akzeptierten Wellentheorie
des Lichtes. Die Beobachtung, dass die Farbe, und damit die Wellenl\"ange bzw.
Frequenz, die Energie der Elektronen bestimmte, war v\"ollig unverst\"andlich.
Die Probleme konnten erst durch die im Jahr 1905 von  Einstein  vorgeschlagene
quantenmechanische Deutung des Photoeffektes beseitigt werden.

