\usepackage{gensymb}
\usepackage[T1]{fontenc}
\usepackage{textcomp}
\usepackage{booktabs,caption,fixltx2e}
\usepackage[flushleft]{threeparttable}
\usepackage{float}
\usepackage{wrapfig}
\usepackage{amsmath}
\usepackage[backend=bibtex,bibstyle=numeric,sorting=nyt]{biblatex}
\usepackage[toc,page]{appendix}
\usepackage{siunitx}
\usepackage{subcaption}
\usepackage{pdfpages}
\sisetup{output-exponent-marker=\ensuremath{\mathrm{E}}}
\usepackage{dirtytalk}
\usepackage{pgfplots}
\usepackage{tikz}
\usetikzlibrary{arrows}
\usetikzlibrary{patterns}
\usetikzlibrary{plotmarks}
\pgfplotsset{compat=1.11}

% For python code
\usepackage{listings}
\usepackage{color} %red, green, blue, yellow, cyan, magenta, black, white
\definecolor{mygreen}{RGB}{28,172,0} % color values Red, Green, Blue
\definecolor{mylilas}{RGB}{170,55,241}
\lstset{language=Python,%
    xleftmargin=0.5cm,%
    basicstyle=\smaller,%
    breaklines=true,%
    morekeywords={matlab2tikz},
    keywordstyle=\color{blue},%
    morekeywords=[2]{1}, keywordstyle=[2]{\color{black}},
    identifierstyle=\color{black},%
    stringstyle=\color{mylilas},
    commentstyle=\color{mygreen},%
    showstringspaces=false,%without this there will be a symbol in the places where there is a space
    numbers=left,%
    numberstyle={\tiny \color{black}},% size of the numbers
    numbersep=9pt, % this defines how far the numbers are from the text
    emph=[1]{for,end,break},emphstyle=[1]\color{red}, %some words to emphasise
    %emph=[2]{word1,word2}, emphstyle=[2]{style},    
}

