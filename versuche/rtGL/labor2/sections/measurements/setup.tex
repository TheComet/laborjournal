\subsection{Measurement Setup}

The setup consisted of 3 main components.

\begin{itemize}
    \item Plant
    \item Controller
    \item PC with the user interface running under MATLAB
\end{itemize}

The plant itself consisted of the following elements.

\begin{itemize}
    \item \textbf{Motor}: The motor used in the setup was a universally usable DC motor. The brushed DC motor is excited using permanent magnets.
    \item \textbf{Tachometer}: The speed sensor which is physically connected to the motor is a tachometer generator. It yields a voltage signal that is proportional to the motor speed.
\end{itemize}

The controller consisted of an Arduino UNO  microcontroller along with a motor
shield (Adafruit Motor Shield v2.3) and a  voltage  divider circuit. The motor
shield is  based  on  an  H  bridge  and is capable of delivering the required
current (as opposed to the Arduino UNO board itself).  The  voltage divider is
required in order to transform the tachometer  signal into a signal compatible
with the analog input.

Microcontroller and PC are connected with a  USB  cable and communicate across
the serial port.

Additionally,  the  laboratory  setup  includes  a  circuit  that  allows  the
simulation of an external  disturbance  (e.g.  a  sudden  increase of the load
driven by the motor) by changing the position of a switch.

The  rpm could be changed by an input control voltage ranging from  $-13V$  to
$13V$.

