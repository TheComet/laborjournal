\subsection{Least Squares Fit}

Given a discrete set of data points which represent a measured step  response of
an  unknown system, the input parameters  of  the  function  $G_n(s,r)$  can  be
tweaked such that the squared error between its step response and the input data
is minimised. The squared error is computed using:

\begin{equation}
    S = \sum_{i} \left(g(t_i)-y_i\right)^2
\end{equation}

where $g(t)$ is the inverse Laplace transform  of $\frac{G_n(s,r)}{s}$ (the time
domain step response) and ($t_i$,$y_i$) are  data  points  of  the measured step
response.

By fitting a system $G_n(s,r)$ to  the  input  data,  it  is possible to further
refine the results  obtained by the two methods mentioned thus far, or otherwise
find  optimal  values  for  $T$  and  $r$ when dealing with  noisy  input  data.

The possibility of finding local minima exists. It is therefore advised to first
use one of the four previous methods to find optimal initial values for $T$, $r$
and $n$ before performing the fit.

It will be shown that the  least  squares  fit  approach  will  yield  the  most
accurate results by orders of magnitude. The downside to this method, of course,
is the large amount of computation time required.
