\section{Diskussion}

In  diesem  Bericht wurde ein ziemlich Leistungsschwacher Motor gew\"ahlt.  Es
hat  schon  etwa  \SI{5}{\second}-\SI{10}{\second}  bis der Sollwert  erreicht
wird.

Die  gr\"ossten  limitierenden  Faktoren sind  das  \"Ubersetzungsverh\"altnis
$c\phi$ und die Strom\"anderung durch  den  Ankerkreis  (limitiert  durch  den
Ankerwiderstand  $R_a$  und  Ankerinduktivit\"at  $L_a$).  Je  gr\"osser  bzw.
schneller  diese  Gr\"ossen  sind,  desto  schneller  kann die  Regelung  auch
stattfinden.

Die  \SI{20}{\milli\second}  Abtastperiode l\"asst auch  nicht  unbedingt  ein
schnelles   Regeln   zu,   aber  f\"ur  diese  Simulation  hat  es   gereicht.

F\"ur  leistungsst\"arkere Motoren w\"urde man aber keine  Gleichstrommaschine
mehr nehmen, sondern man w\"urde in Richtung Drehstrom-Synchronmaschine gehen.
Diese werden anders  betrieben  und  es w\"are interessant, diese Simulationen
nochmals  durchzuf\"uhren  aber  mit  modellierung  der  Br\"uckenschaltungen.

Simscape,  Simulink  und MATLAB machen es unglaublich einfach, solche  Systeme
aufzubauen  und  zu  testen.  Die Theorie, die dahinter steckt, muss man dabei
auch  gar  nicht  verstehen  (aber  es  hilft nat\"urlich, wenn man es weiss).

Die  Simulationsdaten  und   die   Source-Files   des  Berichts  k\"onnen  auf
GitHub\cite{ref:TheComet93} heruntergeladen werden.
