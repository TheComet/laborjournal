\section{Einf\"uhrung}

F\"ur viele  Applikationen  ist  das  genaue  Regeln der Drehzahl eines Motors
wichtig,  unabh\"angig  von  der  Last.  Beispiele  sind  Lokomotiven,  Autos,
CNC Maschinen,  Generatoren  und  so  weiter.  Eine  Lokomotive  sollte  nicht
langsamer bergauf fahren, ein Fr\"aser sollte nicht  bei dichteren Materialien
langsamer fr\"asen und die Netzfrequenz eines elektrischen  Generators  sollte
nicht abnehmen, wenn mehr Ger\"ate Strom konsumieren.

Die L\"osung dieses  Problems liegt in der Regelungstechnik. Der Motor kann in
ein  Regelkreis gesetzt werden und von einem Regler gespiesen werden,  welcher
die   \textit{Ist}-Drehzahl  mit  einer   \textit{Soll}-Drehzahl   vergleicht.

In dieser Arbeit wird ein solches System mit Hilfe  von \textit{Simscape}  und
\textit{Simulink} modelliert, simuliert, und optimiert.

