\section{Diskussion}

Die verwendete Methode ist sehr spezifisch abgestummen worden f\"ur die kleine
Anzahl Bilder, die wir zur Entwicklung verwendeten. W\"urden  andere  Bauteile
die Farbe Blau  enthalten,  oder  w\"are  die  Leiterplatine  Blau  oder  hoch
reflektiv, oder w\"aren die Widerst\"ande  \textit{nicht}  Blau sondern Beige,
oder  w\"urde  man  das Bild mit anderen  Lichtverh\"altnissen  aufnehmen,  so
w\"urde die Erkennung nicht mehr funktionieren.

Es ist uns nicht gelungen, die  Farben  perfekt  zu dekodieren, vorallem nicht
bei \"ahnlichen Farben  wie Rot und Braun, oder Grau und Silber, oder Gold und
Gelb.  Es  w\"are  m\"oglich, dies bei der Dekodierung entgegenzuwirken, indem
``unm\"ogliche'' Farb-Reihenfolgen detektiert werden und  von  der  Gruppe von
m\"oglichen Ergebnissen eliminiert werden. Die Zeit daf\"ur hat gefehlt.

Ein anderer Ansatz mit neuralen Netzwerken wurde betrachtet,  aber  wegen  der
unmengen  n\"otiger  Bilder  zur  Trainierung  des Netzwerkes nicht umgesetzt.

Der  MATLAB-Code kann online auf GitHub\cite{ref:matlab-code} angeschaut resp.
heruntergeladen werden.
