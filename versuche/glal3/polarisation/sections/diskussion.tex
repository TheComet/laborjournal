\section{Resultate und Diskussion}

Die  berechneten  Werte  sind  als  Tabelle  nochmals  zusammengefasst  (Tabelle
\ref{tab:zusammenfassung}).

\begin{table}[ht!]
    \begin{center}
        \caption{Zusammenfassung der berechneten Werte}
        \label{tab:zusammenfassung}
        \begin{tabular}{ll}
            \toprule
            \midrule
            Intensit\"at $I_0$        & $152.541 \pm 19.739$ \SI{}{\lumen\per\square\meter} \\
            Offset $\varphi_{offset}$ & $4.7593  \pm 0.0421$ \SI{}{\degree} \\
            \bottomrule
        \end{tabular}
    \end{center}
\end{table}

Das in  der  Theorie  beschriebene  Modell scheint mit den gemessenen Werten gut
\"ubereinzustimmen. Es wurde ein systematischer  Fehler  beim  Messen des ersten
Versuchs entdeckt, welche bei der Fehlerrechnung ber\"ucksichtigt werden konnte.

Die zwei Python-Programme, welche  die  theorie  visualisiert haben, halfen beim
Verst\"andnis sehr. Ich habe beim Aufarbeiten  der Theorie auch ein tolles Video
gefunden, der das ganze sehr visuell erkl\"art.\cite{ref:polarisation-video}

