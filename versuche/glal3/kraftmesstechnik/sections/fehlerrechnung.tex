\section{Fehlerrechnung}

\subsection{Gleitreibungskraft}

Bei   der  ersten  Messung,  wo  die  Geschwindigkeit  konstant  war   und   die
Gleitreibungskraft  bei  verschiedenen  Gewichten  gemessen  wurde  (Abbildungen
\ref{fig:teppich_const_geschwindigkeit}                                      und
\ref{fig:plastik_const_geschwindigkeit}), wurden die Messpunkte  mit  der Formel
$m_{gl} = \micro_{gl} \cdot m$ gefittet, wobei $m$  die  Masse  des Objektes ist
(X-Achse) und $m_{gl}$ der  Gleitreibungsgewicht,  die  vom  Messger\"at  direkt
abgelesen werden konnte (Y-Achse). Es kann abgelesen werden:

\begin{align*}
    \micro_{gl,\textrm{teppich}_1} &= (226.374 \pm 2.653)\num{e-3} \\
    \micro_{gl,\textrm{plastik}_1} &= (163.187 \pm 0.779)\num{e-3}
\end{align*}

Bei der zweiten Messung, wo der Gewicht konstant war und  die Gleitreibungskraft
bei    verschiedenen     Geschwindigkeiten     gemessen    wurde    (Abbildungen
\ref{fig:teppich_const_gewicht} und \ref{fig:plastik_const_gewicht}), wurden die
Messpunkte   mit  einer   Konstante   $N$   gefittet.   Diese   entspricht   des
durchschnittlichen Gleitreibungsgewichts (mitsammt Fehler). 

Der  Gleitreibungskoeffizient $\micro_{gl_2}$  ist  Geschwindigkeitsunabh\"angig.

\begin{align*}
    \micro_{gl,\textrm{teppich}_2} &= \frac{N_{teppich}}{m} = \frac{(943.750 \pm 18.218)\SI{}{\gram}}{\SI{4000}{\gram}} = (235.938 \pm 4.555)\num{e-3} \\
    \micro_{gl,\textrm{plastik}_2} &= \frac{N_{plastik}}{m} = \frac{(687.500 \pm 14.970)\SI{}{\gram}}{\SI{4000}{\gram}} = (171.875 \pm 3.743)\num{e-3}
\end{align*}

Die Gleitreibungskoeffizienten  der  beiden  Messungen  k\"onnen  dann gewichtet
gemittelt werden:

\begin{align*}
    \overline{\micro_{gl}} &= \frac{\sum_i g_{\overline{\micro i}} \cdot \micro_i}{\sum_i g_{\overline{\micro i}}} \hspace{5mm}\textrm{mit}\hspace{5mm} g_{\overline{\micro i}} = \frac{1}{s_{\overline{\micro i}}} \\
    s_{\overline{\micro_{gl}}} &= \frac{1}{\sqrt{\sum_i g_{\overline{\micro i}}}}
\end{align*}

Es ergeben sich die Gleitreibungskoeffizienten:

\begin{align*}
    \micro_{gl,\textrm{teppich}} &= 0.2276 \pm 0.0015 \\
    \micro_{gl,\textrm{plastik}} &= 0.1636 \pm 0.0009
\end{align*}


\subsection{Grenzhaftkraft}

Die   Grenzhaftkraft  wurde  an  vier  verschiedenen  Stellen   der   Bahn   mit
verschiedenen  Zusatzgewichten gemessen. Die vier Stellen wurden gemittelt  (mit
fehler)   und   mit   QtiPlot   wurden   diese  Mittelwerte   linear   gefittet.

Die Haftkoeffizienten k\"onnen jeweils direkt aus den Fits gelesen werden. Diese
sind:

\begin{align*}
    \micro_{H,\textrm{teppich}_{0.4}} &= (294.967 \pm 5.470)\num{e-3} \\
    \micro_{H,\textrm{teppich}_{3.0}} &= (336.778 \pm 9.133)\num{e-3} \\
    \micro_{H,\textrm{plastik}_{0.4}} &= (294.910 \pm 3.943)\num{e-3} \\
    \micro_{H,\textrm{plastik}_{3.0}} &= (284.750 \pm 2.884)\num{e-3}
\end{align*}

Die Koeffizientenpaare,  die  aus  den  verschiedenen  Geschwindigkeitsmessungen
entstanden  sind,  k\"onnen  weiter  gewichtet gemittelt werden. Es ergeben sich
somit die Grenzhaftkoeffizienten:

\begin{align*}
    \micro_{H,\textrm{teppich}} &= 0.3106 \pm 0.0018 \\
    \micro_{H,\textrm{plastik}} &= 0.2890 \pm 0.0013
\end{align*}

