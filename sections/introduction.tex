\section{Introduction}

The  need  to  identify the dynamic behaviour of an unknown, arbitrary system or
\textit{plant}, as it is referred to in control theory, is a common problem. The
plant's behaviour dictates  how  the \textit{controller} must behave in order to
create  a  stable  \textit{control  loop}.  The  ability to  accurately  and  --
preferrably  automatically  -- deduce this behaviour is crucial. Small errors in
the identification process can lead to large deviations later on. Worse, it  can
lead to unstable control loops.

There  are  many  ways  to  measure  and  identify  a  system and there are many
different  classes  of systems to be measured. In this report  we  will  confine
ourselves to linear, time-invariant  (LTI)  systems  only. Additionally, we will
also  confine ourselves to systems  that  \textbf{don't}  exhibit  overshoot  or
undershoot.

The focus of this report will be the comparison of two fairly similar methods of
system   identification   proposed   by    L.    Sani\cite{ref:sani}    and   P.
Hudzovic\cite{ref:hudzovic}.

Furthermore,  an extension to both methods involving a least squares fit will be
proposed,  allowing  the  results  of   both  methods  to  be  further  refined.

