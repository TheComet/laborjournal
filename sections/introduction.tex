\section{Introduction}

The  need  to  identify the dynamic behaviour of an unknown, arbitrary system or
\textit{plant}, as it is referred to in control theory, is a common problem. The
plant's behaviour dictates  how  the \textit{controller} must behave in order to
create  a  stable  \textit{control  loop}.  The  ability to  accurately  and  --
preferrably  automatically  -- deduce this behaviour is crucial. Small errors in
the identification process can lead to large deviations later on. Worse, it  can
lead to unstable control loops.

The focus of this report will be  the  identification  of  time-invariant  (LTI)
systems that don't exhibit overshoot  or  undershoot.  specifically,  two fairly
similar methods  of system identification proposed by L. Sani\cite{ref:sani} and
P. Hudzovic\cite{ref:hudzovic} will be  analysed  and  compared  to one another.
 
Furthermore,  an extension to both methods involving a least squares fit will be
proposed,  allowing  the  results  of   both  methods  to  be  further  refined.

