\subsection{The Model}

The  model  used  by  both  Hudzovic\cite{ref:hudzovic}  and Sani\cite{ref:sani}
approximate  the  step  response of a plant by using a series  of  PT1  elements
multiplied  together with varying time constants $T_k$ to form  a  PTn  element,
$G(s)$. This is defined as:

\begin{equation}
    G_n(s,r) = y_0 + K_s \prod_{k=1}^{n} \frac{1}{1+s \cdot T_k(r)}
    \label{eq:ptn}
\end{equation}

where the scale factor, $K_s$, is defined as:

\begin{equation}
    K_s = \frac{xa(\infty)}{xeo}
\end{equation}

The  transfer function in equation \ref{eq:ptn} serves as a basis to  model  the
step response of many systems.

Rather than individually having to find the time constants  $T_1\ldots  T_n$  --
the effort of which would greatly increase with the order $n$ -- the two methods
of  P.  Hudzovic  and  L.  Sani instead calculate these constants using a common
function $T_k(r)$.

The  approach  proposed  by  P.  Hudzovic\cite{ref:hudzovic}  for  $T_k(r)$  is:

\begin{equation}
    T_k(r) = \frac{T}{1 - (k-1)r}
    \label{eq:hudzovic}
\end{equation}

where  the  constant  $r$ must be  confined  to  the  interval  $0  \le  r  \leq
\frac{1}{n-1}$.

The   approach   proposed   by   L.   Sani\cite{ref:sani}   for   $T_k(r)$   is:

\begin{equation}
    T_k(r) = T \cdot r^k
    \label{eq:sani}
\end{equation}

where the constant $r$ must be confined to the interval $0 \leq r \le 1$.

As  can  be  seen,  in both cases, the  problem  has  been  reduced  to  finding
appropriate values for  $n$,  $T$  and  $r$  such  that  the  step  response  of
$G_n(s,r)$ approximates the data acquired from the plant as closely as possible.

