\subsection{Hudzovic vs Sani}

In this section the  two methods of approximating a step response proposed by P.
Hudzovic\cite{ref:hudzovic}  and  L.  Sani\cite{ref:sani}  will   be  discussed.


\subsubsection{Complexity of Step Responses}

It is important to realise that no matter how a step response function is scaled
or  offset  (defined  by the parameters $K_s$ for amplitude, $T$ for time scale,
$x_0$ and $y_0$ for offset) the actual \textit{shape}  always  remains the same.
Thus, only its shape tells us something about its complexity.

Therein lies the key to characterising a step response. A method for determining
how  ``simple''  or how ``complex'' a step response is needs to  be  devised  --
independent of scale and offset -- before it is possible to start modelling  and
fitting a system to it. The reason why we need to know  this ``complexity'' will
become clear later.

\begin{figure}
    \includegraphics[width=\linewidth]{images/tu_tg}
    \caption{Method of P. Hudzovic, determine Tu and Tg}
    \label{fig:tu_tg}
\end{figure}

P. Hudzovic proposed the following method (see figure \ref{fig:tu_tg}):
\begin{itemize}
    \item
Find  the  point  of  inflection of the step function. This  typically  involves
calculating the derivative and searching for a maximum.
    \item
Place a tangend in said points and find the intersections  with  the minimum and
maximum horizontal lines.
    \item
The distance between the two intersections  is  referred  to  as  $T_g$, and the
distance between  the minimum intersection point and the beginning of the signal
is referred to as $T_u$.
\end{itemize}
The ``complexity'' of the  step  response  is  defined by the ratio of $T_u$ and
$T_g$ and is written as:

\begin{equation}
    \textrm{plant}_{Tu/Tg} = \frac{T_u}{T_g}
    \label{eq:tu_tg}
\end{equation}

\begin{figure}
    \includegraphics[width=\linewidth]{images/t10_t50_t90}
    \caption{Method of L. Sani, determine t10, t50 and t90}
    \label{fig:t10_t50_t90}
\end{figure}

L.  Sani  proposed   a  different  method  (see  figure  \ref{fig:t10_t50_t90}):
Determine the  times  $t_{10}$,  $t_{50}$ and $t_{90}$ required for reaching the
values   at    \SI{10}{\percent},    \SI{50}{\percent}    and   \SI{90}{percent}
respectively.

The   ``complexity''   of  the  step  response  is  defined  by  the  ratio   of
$t_{90}-t_{10}$ and $t_{50}$, otherwise referred to as $\lambda$, and is defined
as:

\begin{equation}
    \textrm{plant}_{\lambda} = \frac{t_{90}-t_{10}}{t_{50}}
    \label{eq:t10_t50_t90}
\end{equation}

Visually, one can  see how decreasing $T_g$ in figure \ref{fig:tu_tg} causes the
step response to become steeper (i.e. it becomes more ``complex''). The value of
$\textrm{plant}_{T_u/T_g}$  in  equation  \ref{eq:tu_tg}  increases.  Similarly,
decreasing the difference $t_{90}-t_{10}$  in  figure \ref{fig:t10_t50_t90} also
causes   the   step   response   to   become    steeper   and   the   value   of
$\textrm{plant}_{\lambda}$ increases.

On ther other hand, one can also see how increasing $T_u$ and $t_{50}$ increases
the  lag time of the step response, which also leads to a higher ``complexity''.

We will later see how the complexity relates to the required  order  $n$  of the
system used to model the step response.


\subsubsection{Approximating the Step Response}

The  approach  of   both   Hudzovic\cite{ref:hudzovic}  and  Sani\cite{ref:sani}
approximate  the  step  response  of a plant by using a series of  PT1  elements
multiplied together with varying time constants  $T_k$  to  form  a PTn element,
$G(s)$. This is defined as:

\begin{equation}
    G_n(s,r) = y_0 + K_s \prod_{k=1}^{n} \frac{1}{1+s \cdot T_k(r)}
    \label{eq:ptn}
\end{equation}

where the scale factor, $K_s$, is defined as:

\begin{equation}
    K_s = \frac{xa(\infty)}{xeo}
\end{equation}

Rather than individually having to find the time constants  $T_1\ldots  T_n$  --
the effort of which would greatly increase with the order $n$ -- the methods  of
P.  Hudzovic  and  L.  Sani  instead  calculate  these constants using a  common
function  $T_k(r)$.  P.  Hudzovic's  approach\cite{ref:hudzovic} for $T_k(r)$ is
defined as:

\begin{equation}
    T_k(r) = \frac{T}{1 - (k-1)r}
    \label{eq:hudzovic}
\end{equation}
where  the  constant  $r$ must be  confined  to  the  interval  $0  \le  r  \leq
\frac{1}{n-1}$.

Whereas L. Sani's approach\cite{ref:sani} defines $T_k(r)$ as:

\begin{equation}
    T_k(r) = T \cdot r^k
    \label{eq:sani}
\end{equation}
where the constant $r$ must be confined to the interval $0 \leq r \le 1$.

The problem, thus, is to find appropriate  values for $n$, $T$ and $r$ such that
the step  response  of  $G(s)$  approximates the data acquired from the plant as
closely   as   possible.

Unfortunately,  it is not possible to directly calculate appropriate values  for
$n$, $T$ and $r$, however,  what  equations  \ref{eq:ptn}, \ref{eq:hudzovic} and
\ref{eq:sani}  allow  us  to do now is calculate  an  infinite  number  of  step
responses of arbitrary  order  (while  maintaining  a  constant  number of input
arguments,  thanks  to  the  function  $T_k(r)$), normalise the  response  using
equations \ref{eq:tu_tg}  or  \ref{eq:t10_t50_t90}, and perform a reverse lookup
on those  results  to  find the parameters $n$, $T$, and $r$. In practice, it is
sufficient to calculate about 50 step responses for each order  and  interpolate
between those values  when  doing  the  lookup,  as will be shown in this paper.

As discussed earlier, a remarkable  observation  is  that the parameters $r$ and
$n$ are independent of time and amplitude (and offset); that is,  the normalised
step response does not change its  shape when the parameter $T$ is changed. This
is fantastic,  because  it  allows  us  to eliminate a dimension from the lookup
table.

If we set $T=1$, $K_s=1$ and $y_0=0$, we  can  use  equations  \ref{eq:ptn}  and
\ref{eq:hudzovic} to  calculate  a  series  of  transfer  functions  $G_n(s,r)$,
calculate  their  time  domain  step  responses   $g_{r,n}(t)$,   and  calculate
$g_{T_u/T_g}$ and $g_{1/T_g}$ for each step response.

Thus, $g_{T_u/T_g}$ and $g_{1/T_g}$ can both be expressed as  functions  of  $r$
and $n$:

\begin{align}
    g_{T_u/T_g} = g_{T_u/T_g}(r, n) \\
    g_{1/T_g}  =  g_{1/T_g}(r,n)
\end{align}

\begin{figure}
    \includegraphics[width=\linewidth]{images/hudzovic_curves}
    \caption{wtf}
    \label{fig:hudzovic}
\end{figure}

Visualising these functions yields the curves seen in figure \ref{fig:hudzovic}.
What these plots show beautifully  is that the higher the order $n$, the steeper
-- or more  ``complex''  --  the  step response becomes (smaller values of $T_g$
mean faster rise times of the step responses).
Another important thing to observe is how lower orders of $G_n(s,r)$ aren't able
to  rise  as  fast as higher orders are able to, \textbf{regardless} of $r$  and
$T$. This can  also  be  seen  in figure \ref{fig:hudzovic}, lower orders cannot
reach ratios of $T_u/T_g$ that higher orders can.


As discussed earlier, determining the  complexity of a step response is directly
related  to  the  requried  order $n$ of the model. Finding $n$ is now a  simple
matter of  computing $\textrm{plant}_{T_u/T_g}$ and comparing the result to each
curve in \ref{fig:hudzovic} for $r=0$. Ideally, we'd like  $n$ to be as small as
possible, so the following equation must be satisfied.

\begin{equation}
    g_{T_u/T_g}(r, n)\lvert_{r=0} \hspace{2mm} \le \hspace{2mm} plant_{T_u/T_g}
\end{equation}

With the  parameter $n$ defined, the next step is to find the intersection point
of  $g_{T_u/T_g}(r, n)$ with the horizontal line  $plant_{T_u/T_g}$.  This  will
yield parameter $r$.

The last parameter, $T$,  can  finally  be  determined  by evaluating $T_g \cdot
g_{1/T_g}(r, n)$. Graphically, this equates to finding the intersection point of
the vertical line going through $r$ and the function $g_{1/T_g}(r, n)$ in figure
\ref{fig:hudzovic} and multiplying the result by $T_g$.

\begin{figure}
    \includegraphics[width=\linewidth]{images/sani_curves}
    \caption{wtf}
    \label{fig:sani}
\end{figure}

