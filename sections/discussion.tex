\section{Discussion}

The two methods proposed by L. Sani and P. Hudzovic and two combinations thereof
were investigated and compared to one another.

The most accurate method -- in general -- turns out to be P. Hudzovic's transfer
function in conjunction with L. Sani's  characterisation  method  (by  measuring
$t_{10}$, $t_{50}$, $t_{90}$ instead of $T_u/T_g$). This  is true for all orders
of  $n$  and  this  is  true  for   noisy   and   non-noisy   input   functions.

Of course,  by performing a least-square curve fit, an even more accurate result
can be obtained.

One factor  that was not considered was how many data points are required in the
lookup curves to yield accurate results. The simulations in this  report used 50
data  points.  It  would  be  worth investigating this further to see how higher
resolution lookup curves affect the accuracy of each method.

P.  Hudzovic's  characterisation method (by measuring $T_u/T_g$) isn't as robust
as  L.  Sani's   characterisation   method  (by  measuring  $t_{10}$,  $t_{50}$,
$t_{90}$),  especially  with  noisy  input data. This is primarily  due  to  the
derivative  that  must  be  computed  to   find   the   point   of   inflection.

L. Sani's method is definitely  the  easiest  to  implement  due  to  the simple
interpolation formulae and due to  the simplicity of finding $t_{10}$, $t_{50}$,
$t_{90}$. It also performs quite  fast  and has a low memory footprint (since it
doesn't require any lookup curves).  Unfortunately, it does perform the worst of
all methods.

