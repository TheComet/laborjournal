\documentclass[notitlepage]{fhnwreport/fhnwreport}

\usepackage{gensymb}
\usepackage{textcomp}
\usepackage[T1]{fontenc}
\usepackage{lipsum}
\usepackage{titling}       % Abstract on titlepage
\usepackage{changepage}    % Change width of abstract text
\usepackage{float}
\usepackage{amsmath}
\usepackage{amssymb}
\usepackage{mathrsfs}
\usepackage{todonotes}
\usepackage[backend=bibtex,bibstyle=numeric,sorting=nyt]{biblatex}
\usepackage[toc,page]{appendix}
\usepackage{siunitx}
\usepackage{subcaption}
\usepackage{pdfpages}
\sisetup{output-exponent-marker=\ensuremath{\mathrm{E}}}

% Labels per section
\numberwithin{equation}{section}

% For MATLAB code
\usepackage{listings}
\usepackage{color} %red, green, blue, yellow, cyan, magenta, black, white
\definecolor{mygreen}{RGB}{28,172,0} % color values Red, Green, Blue
\definecolor{mylilas}{RGB}{170,55,241}

% Also required for tuning appearance of MATLAB code
\usepackage{microtype}
\usepackage[scaled]{beramono}
\newcommand\smaller{\fontsize{8}{8.2}\selectfont}
\newcommand*\LSTfont{\smaller\ttfamily\SetTracking{encoding=*}{-60}\lsstyle}

% Laplace L
\DeclareMathOperator{\laplace}{\mathscr{L}}



\begin{document}
\section*{Zusammenfassung Leistungsberechnung bei nicht sinus-f\"ormigen Gr\"ossen (Power Elektronik)}
\hfill \textit{Alex Murray}

In aet1  und  aet2  hatten  wir  gelernt,  die  Blindleistung,  Wirkleistung und
Scheinleistung  in  Funktion  von  Strom-  und  Spannungsspitzenwert  und  deren
Phasenverschiebung  $\varphi$  zu  berechnen.  Wir  hatten  auch  gelernt,  dass
$\cos\varphi$ der Leistungsfaktor ist.

Dies  funktionierte  nur  mit  der  Annahme,  dass   Strom   und  Spannung  rein
sinusf\"ormige Funktionen waren. Bei der Leistungselektronik ist  dies  h\"aufig
nicht  mehr  der  Fall.  Baugruppen  wie   zum   Beispiel   Gleichrichter   oder
Schaltwandler  verzerren Strom und Spannung, sodass Oberschwingungen  entstehen.

Die Phasenverschiebung kann schlecht  definiert  werden.  Man spricht nicht mehr
von  $\cos\varphi$,  sondern  vom  Leistungsfaktor  $\lambda$,  welcher  aus dem
Verh\"altnis $\frac{Q}{P}$  berechnet  wird. Die Berechnung von $Q$ und $P$ muss
neu hergeleitet werden.

In  diesem Versuch geht es um die Leistungsberechnung von  nicht-Sinusf\"ormigen
gr\"ossen. Es  wird  auch  behandelt,  wie  anhand  der  Fourierreihe ein Signal
zerlegt werden kann, um die Leistung in eimen Frequenzabh\"angigen Widerstand zu
berechnen.

In  der  ersten  Aufgabe  sind  zwei   periodische   Signale   vorgegeben,   ein
Spannungssignal und  ein  Stromsignal.  Der  lineare  Mittelwert wird berechnet,
indem   man   das   Signal   \"uber  eine  Periode  integriert  und  durch   die
Periodenl\"ange  teilt.  Bei  einem  sinusf\"ormigen  Signal w\"urde aus  dieser
Berechnung  nat\"urlich  Null  resultieren.  Der  Effektivwert   wird  \"ahnlich
berechnet: Das \textit{Quadrat} des Signals wird \"uber eine Periode integriert,
durch die Periodenl\"ange geteilt, und dann wird die  Wurzel  gezogen. Bei einem
Sinusf\"ormigen   Signal   w\"urde   aus  dieser  Berechnung  der   wohlbekannte
Faktor $\frac{1}{\sqrt{2}}$ entstehen.

Weil  das  Stromsignal  einweggleichgerichtet  wurde  (vermutlich),  entsand aus
dieser Berechnung auch der  Faktor  $\frac{1}{\sqrt{2}}$, nur durch zwei geteilt
weil die H\"alfte des Signals fehlt.

Als  n\"achstes mussten die Blind-, Wirk-, und Scheinleistungen  allgemein  (und
somit auch der Leistungsfaktor $\lambda$) in Funktion der Verschiebung  $\theta$
berechnet werden.

Die Scheinleistung ist  unabh\"angig  von  $\theta$  und  l\"asst  sich  aus den
Effektivwerten  von  Strom  und  Spannung  berechnen.

Die Wirkleistung wird berechnet, indem die Momentanleistung  ($u(t)\cdot  i(t)$)
\"uber eine Periode integriert  wird  und  durch  die Periodendauer geteilt wird
(arithmetischer Mittelwert). In diesem Fall ergibt sich eine Cosinusfunktion mit
Offset.  Die  Wirkleistung  wird  bei  einem $\theta$  von  exakt  einer  halben
Periodendauer  ($\SI{180}{\degree}$)  Null.   Der   Maximalwert  entspricht  der
Leistung der ersten Harmonischen.

Die Oberwellen  des  Rechtecksignals haben keinen Einfluss auf die Wirkleistung.
Dies  zeigt,  dass  die  Wirkleistung  bei   Signalen  mit  Oberwellen  nie  der
Scheinleistung   entspricht    und    immer    Blindleistung    vorhanden   ist.

Die  Blindleistung wird nach wie vor mit der Formel von Pythagoras aus  $S$  und
$P$  berechnet.  Der  Leistungsfaktor  ist   bei   $\theta=\SI{0}{\degree}$   am
gr\"ossten und bei $\theta=\SI{180}{\degree}$ ist sie Null.

F\"ur  die  Berechnung  der  Fourierkoeffizienten  $a_k$  und  $b_k$  wurden die
Integralgrenzen  so  gew\"ahlt,  dass   die  Funktion  gerade  wird.  Die  $b_k$
Koeffizienten  einer  geraden Funktion sind Null. F\"ur die Periodendauer  wurde
$2\pi$ gew\"ahlt. Die Fourierkoeffizienten des Stromsignals wurden  unabh\"angig
von der Verschiebung berechnet.

Aus  den  Fourierkoeffizienten  lassen  sich  die  Amplitude und die  Phase  der
Oberwellen berechnen.  Dabei  k\"onnen  die  koeffizienten  $a_k$  und  $b_k$ zu
$\underline{c}_k$  (komplexe  Zahl)  umgerechnet  werden  um sich das  Leben  zu
vereinfachen.   Dabei  ist  aufgefallen,  dass  die  Amplitude   der   einzelnen
Harmonischen  unabh\"angig  von   $\theta$   ist.   Die   Phase   jedoch  nicht.

Im zweiten  Teil  des  Versuchs  ist  ein  Stromverlauf gegeben, der durch einen
frequenzabh\"angigen  Widerstand fliesst. Die Frequenzabh\"angigkeit  ist  durch
den  Skineffekt  gegeben.

Der Effektivwert des  Stromes  wird  zuerst  nach  der  ``klassischen  Methode''
berechnet. Der erhaltene Wert gilt  nat\"urlich  nur  bei frequenzunabh\"angigen
Widerst\"anden.  Nach der ``richtigen'' Methode muss die  die  Fourierreihe  des
Stromes berechnet werden. Da in diesem Fall die Funktion wieder gerade ist, sind
alle $b_k$ Koeffizienten Null. Zur  Berechnung des Leistungsspektrums wird f\"ur
jede  Harmonische  der Widerstand ausgerechnet. Da  das  Amplitudenspektrum  die
Spitzenwerte anzeigt m\"ussen die Werte in Effektivwerte umgerechnet werden (mit
Faktor $\frac{1}{\sqrt{2}}$ skalieren). Im Leistungsspektrum ist  gut  zu sehen,
dass die erste  Oberwelle  die gr\"osste Leistung liefert. Die Werte streben bei
steigender Ordnungszahl schnell nach Null.

Im Vergleich zur ``klassischen Methode'' ist der  ``richtige'' Effektivwert etwa
um den Faktor $4$ gr\"osser. Die Spannung  \"uber  dem  Widerstand  sieht  recht
verzerrt  aus.  Jeweils  zu  den  Zeiten, bei denen  die  Steigung  des  Stromes
\"andert, gibt es einen grossen Spannungspeak.

F\"ur  die Ermittlung  von  allgemeinen  Formeln  wurde  mupad  eingesetzt.  Die
numerische Berechnung  der  Fourierkoeffizienten  und  die grafische Darstellung
erfolgten mit MATLAB.

In  diesem  Versuch  lernte  ich  zum ersten Mal,  dass  $\cos\varphi$  nur  bei
harmonischen  Spannungs-  und  Stromverl\"aufe  Bedeutung  hat.  Bei  verzerrten
Signalen    muss    auf    die   allgemeine   Methode   mit   Integralrechnungen
zur\"uckgegriffen werden. Ich hatte mich bis jetzt nicht mit Leistungselektronik
auseinandergesetzt, doch ich fand  den  Versuch interessant und ich konnte dabei
Neues lernen.

\end{document}

